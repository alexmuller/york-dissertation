%!TEX root = ../Project.tex

\subsection{Web application frameworks}
\label{sec:webframeworks}

While it is possible to build a web application without using a framework,
there is no advantage to rewriting basic operations for this project. If the
system needed to operate at a scale or speed far beyond what these frameworks
were capable of, it is likely that the software would have to be built from
scratch in order to meet those requirements. With the limited amount of time
allowed for development and implementation (see the project Gantt chart in
Figure~\ref{ganttchart}), it is sensible to spend a small amount of time
choosing a framework and as a result allow more time to implement the unique
parts of this system.

The most important consideration when choosing a framework is the language it
is written in, as this defines, among other things, what kind of functionality
might be offered and how easy the software will be to maintain in the future.

There are three languages used for most of the in-house web development at the
University of York. \Gls{cfml} is a language for creating web pages, the most
popular commercial implementation of which is a commercial one produced by
Adobe. ColdFusion is used widely across the University, including in
applications such as the ``People Directory'' search tool (shown in
Figure~\ref{yorkacuk_directory_search}) and online account management
facilities provided by \gls{itservices}. Java was used in the creation of the
new student portal discussed in Section~\ref{sec:webapps_york}. Finally, a PHP
service is offered by the University for users to deploy their own PHP
applications, though the University states that they do not maintain PHP
applications.

\noindent{\textbf{ColdFusion}}\mynobreakpar

The following is a snippet of ColdFusion that simply outputs the user's
username:

\begin{lstlisting}[language=HTML]
<cfparam name="attributes.userid" type="string" default="unknown">
<h1>Hello, world</h1>
<cfoutput>
  <p>Welcome, #HTMLEditFormat(attributes.userid)#.</p>
</cfoutput>
\end{lstlisting}

However, \gls{cfml} is unsuitable for this project as it seems to be primarily
a markup language -- the relatively complex allocation logic behind this
system would have to be implemented in another language such as Java or
Python. Using two languages to create this application would needlessly
increase the complexity of any future maintenance.

\noindent{\textbf{Java}}\mynobreakpar

While there are a wide variety of Java web application frameworks available,
the University of York has chosen to use Spring, an Apache Software-licensed
framework developed by a division of VMware (a virtualisation software
provider). The Spring framework has a \gls{mvc} component, making it entirely
suitable for this project.

In a presentation at \gls{oscon} in 2007, Matt Raible compared several web
frameworks written in Java; JavaServer Faces (JSF), Spring MVC, Stripes,
Struts 2, Tapestry and Wicket \cite{raible2007javawebframeworks}. Watching the
video of his talk, the striking aspect is how little there is differentiating
all these frameworks. As Spring is a framework that \gls{itservices} have
deployed in the past and are able to support, it makes little sense to choose
anything else. Using Spring also may also allow a significant amount of code
reuse, particularly with regards to integrating with the University's
authentication systems.

\noindent{\textbf{PHP, Ruby, Python or other languages}}\mynobreakpar

Outside the University of York, popular web programming languages include Ruby
and Python. Examples of popular and widely used frameworks for these languages
include CakePHP, Ruby on Rails and Django. There are many other frameworks
available to support building an application in these languages, but
\gls{itservices} do not have the resources on-hand to maintain such an
application and may not even be able to provide a server on which to host it.

\subsubsection{Summary}

The framework chosen for this project is Spring 3, written in Java. As the
application will be maintained by Java developers, it would not be sensible to
implement the software in another language only to possibly have to
re-implement using a different technology stack in the future. This choice of
framework gives the application the best chance of being used by the
University in the future.

PHP, Ruby and Python are all perfectly good languages for web application
development,\footnote{On the popular code-hosting website \gls{github}, Ruby
and Python are ranked 2nd and 3rd respectively in terms of their popularity in
software development (see \url{https://github.com/languages})} but none are
suitable for this project as they cannot easily be hosted or maintained by
\gls{itservices}. Similarly, ColdFusion is unsuitable as it does not have the
necessary features that would be required by the allocation part of the
application.
