%!TEX root = ../Project.tex

\section{Introduction}

% The scope of the project, setting the scene for the remainder of the report.

A paper was presented by \gls{yusu} at \gls{utc} in March 2010 which discussed
student complaints and feedback around the allocation of optional modules
across the University. The issues raised in that paper were supported by
student comments made by final-year undergraduates completing the 2010
\gls{nss}. \gls{utc} established a working group to investigate improving the
module allocation process and this Department of Computer Science project
resulted from that investigation. As such, the project is sponsored by
\gls{utc} and is overseen in that regard by \gls{lc} of the \gls{aso}.

As well as being assessed as an undergraduate project by the Department of
Computer Science, the results of the project will be evaluated independently
by \gls{lc}. If the software created is ultimately judged to provide
sufficient advantages over the current methods of module allocation, it may be
recommended that responsibility for the software is handed to \gls{itservices}
and it is offered to all departments -- though clearly this has other
implications for the University that are outside the scope of this report,
such as resourcing of development and maintenance staff for the application.

\subsection{Report structure}

This section gives background information about the University of York
relevant to the project, including summarising research done by \gls{lc} in
preparation for this project. Section~\ref{sec:requirements} discusses how the
project was run (with the help of administrative staff at the University) and
the requirements that were obtained from the project clients.
Section~\ref{sec:research} covers the research and background reading that was
undertaken at the start of the project. Section~\ref{sec:implementation}
discusses the choices that were made during the implementation of the
software, how it was tested before delivery and the result of the first use by
departments. Section~\ref{sec:furtherwork} describes ways in which the
software could be extended in the future. Section~\ref{sec:conclusions} gives
the conclusions drawn at the end of the project.

\subsection{The current state of module allocation}

% Refer to documents provided by LC

Universities worldwide want to give their students the most interesting and
useful education possible, and one way this can be achieved is by offering as
much flexibility and variation as they can in the modules that students can
take. Very generally speaking, a department that offers a broad range of
modules for students to choose from will see increased student satisfaction
with the course.

At the University of York, course structure and module allocation is dealt
with at the departmental level rather than centrally by the University, though
three departments make use of centrally offered software to allocate modules
on a first-come, first-served basis. Several departments feel the software is
not flexible or usable enough and continue to use paper-based forms that must
be filled out by students and returned to the departmental office. In
departments that use paper forms, the data entry required when allocating
modules is incredibly resource intensive and time consuming. The two pilot
departments involved in this project, Archaeology and History, both use paper
forms -- examples of these are given in Appendix~\ref{sec:paperforms}. A
smaller number of departments have written their own module selection and
allocation software. The Department of Computer Science, for example, has the
technical resources on-hand to maintain this sort of software internally.

At many universities in the United Kingdom (for example Warwick, Leeds, Essex
and Bath), enrolment is completed online, on a first-come, first-served basis.
Other institutions, such as the University of Sheffield and Durham University,
use paper-based systems, but again on a first-come, first-served basis. This
method of allocating modules has its own drawbacks, including students needing
to arrive at their department early and queue in order to submit their module
choices in good time and therefore stand a chance of receiving the modules
they requested. The working group set up by \gls{utc} decided against both
paper-based and first-come, first-served systems as they were perceived to be
time-consuming and potentially unfair.

\subsection{Web applications at the University of York}
\label{sec:webapps_york}

% Student portal, timetabling gateway, Google Apps for Education

As would be expected of any world-class institution, the University of York is
continually improving the quality of web software available to students and
staff through providing updates to commercially obtained software and
employing developers to improve and maintain code written in-house.

For the last few years, a \gls{sso} solution has been provided centrally,
using an open source product called Shibboleth. This improves both user
experience and security, as users are encouraged to only ever enter their
University username and password into a familiar-looking page with a web
address that always begins with \texttt{https://shib.york.ac.uk/}. In
September 2011 a new ``student portal'' was released, allowing students to
view personalised information relevant to them (such as their timetable,
library loans and news) in one location. The application was created by
\gls{itservices} and was written in Java. A screenshot is given in
Figure~\ref{yorkacuk_student_portal}, and further information on the student
portal is available at \texttt{https://www.york.ac.uk/students/about/}. At the
beginning of the 2011--12 academic year, new timetabling software was made
available to give members of the University access to their complete timetable
in one location -- something which has not been possible before. In June 2012,
the University will move all students to Google's Apps for Education product
for email and calendaring. One of the aims of this move is to improve the user
experience dramatically over the current webmail software.

The increasing scope and number of web applications at the University
hopefully gives some indication that students and staff are becoming more and
more familiar with accessing information online. User experience and usability
are a key focus of modern web applications, where perhaps they might have
taken a back seat in the past. As they are exposed to polished applications
from household names like Google, users' tolerance for poorly coded
applications that do not provide the required functionality will only
decrease.

