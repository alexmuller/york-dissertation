%!TEX root = ../Project.tex

\subsection{Sensitive information and the Data Protection Act}
\label{sec:dataprotection}

% As noted in the Statement of ethics, we should discuss CF's input here.

The software created for the purpose of this project will manipulate students'
personal data, and as such the University's Data Protection Officer was
consulted throughout the duration of the project; the requirements gathering,
design and implementation. He provided input on how the software should comply
with all relevant legislation -- this section outlines his input and resulting
actions taken.

\noindent{\textbf{Automated decision making}}\mynobreakpar

Under Section 12 (Part II) of the Data Protection Act 1998 c.29, an individual
(in this case, a student) can request that no automated decisions are made
that affect that individual. The relevant legislation is available online in
full at \url{http://www.legislation.gov.uk/ukpga/1998/29/section/12}. For the
purposes of this project, departments must notify students that their modules
will be allocated automatically -- words to this effect were placed in the
departmental student handbooks for this academic year, and will be reinforced
by email when students are asked to choose their modules.

The project steering group decided that any student who requests that their
modules are allocated manually will be treated as an exception. However, the
departments noted that it would be very hard to ensure every student was
treated fairly when some were allocated using the software and others were
not. It is expected that only a tiny proportion of students would make this
request, if any. As in previous years, students will of course have the right
to appeal their module allocation with the department if they feel it is
unfair. Explanatory text will be placed on the application that explains the
nature of this project and how the software came to be, and students will be
encouraged to contact their departments if they have any concerns.

\noindent{\textbf{Retaining a memory}}\mynobreakpar

On the question of retaining student data so that the application can
implement some kind of memory feature, the Data Protection Officer commented
that this information would need to be held in line with University policies
and a data retention policy would have to be created for this system. However,
this is not especially relevant for the pilot run of this application as no
data will be retained yet. \gls{lc} will note in her evaluation that if the
software is to be used next year, this retention policy will have to be
developed.

\noindent{\textbf{Transferring data between systems}}\mynobreakpar

The Data Protection Act requires any transfer of data between University
systems to be ``secure and accurate''. Some student data (usernames and course
information) will be transferred out of \gls{sits} by departmental
administrators, who will upload it into the application. This fulfills the
criteria that the transfer is secure, as no other party will have access to
either system or any of the data on departmental PCs, and that the transfer is
accurate, as departments will ensure that the correct number of students and
courses are uploaded to the system.

The data transferred into \gls{sits} will be a list of allocations in some
format. This information will be encrypted before being transferred between
the department and \gls{ssdt} to ensure security, and will be sanity-checked
by both the departments and by \gls{ssdt} after it is inserted into
\gls{sits}. Departments can check, for example, that each student has been
given the requisite number of modules to ensure that the transfer was
accurate. They will have to rely on complaints from students to discover that
each student's allocation is accurate -- there is no other way for a
department to verify every student's choices and allocations.

\noindent{\textbf{Compromise of student data}}\mynobreakpar

The Data Protection Officer notes that \emph{any} data input by students into
the system is ``their personal data'', and any compromise of this data would
be a breach of the Data Protection Act. The project implementer will work with
\gls{itservices} during the creation of the software and they will ensure as
far as they can that there are no security issues in the software that would
cause personal information to be compromised. Furthermore, the software will
be restricted by the University firewall so that it can only be accessed by a
user with a valid University of York username and password. While this does
not completely remove the risk of a data breach, it will reduce the impact of
any security problems with the software.
