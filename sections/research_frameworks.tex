%!TEX root = ../Project.tex

\subsection{Web application frameworks}

% Intro to (web) frameworks in general

A web framework is a certain amount of reusable code that helps application
developers by reducing the complexity of implementing common operations. For
example, any moderately complex web application will need to write user input
to a data store, and protecting against malicious input is an obvious concern
when executing code in a database. Many popular frameworks include functions
that write to the database on behalf of the developer, and will automatically
sanitise all input to prevent against attacks. As every application should
sanitise user input, it is this kind of repetitive action that frameworks help
to make easier. A framework benefits the application developer by increasing
the amount of time that can be spent building the unique parts of their
application rather than re-implementing things that have been done time and
time again.

A key focus of \gls{mvc} frameworks is to separate the application logic from
the interface that is presented to the user. The term \gls{mvc} splits the
application into three distinct pieces; the model for structuring and imposing
constraints on the data that the application deals with, the controller for
manipulating the data into a usable format, and the view for presenting the
manipulated data to the end user in an understandable and useful way.

Parr's 2004 paper \cite{Parr2004templateengines} is entirely to do with
separating the model from the view, and he believes that it is something
developers strive for but often fail to achieve. He provides some good
examples of software development patterns that should never be seen, such as
including \gls{sql} statements anywhere except the model -- and even then, the
framework should provide a mapping between application objects and database
structure. He cites maintainability as a reason to strictly separate the view
from both the controller and the model, and as maintainability is a key focus
of this project (see Section~\ref{sec:maintainability}) this is clearly
important.

This intuitively makes sense from a software development perspective; software
that consists of many parts loosely coupled is beneficial to a complex
application with the pieces being difficult to unravel. The web even follows
this same mentality, with the author David Weinberger writing a book titled
``Small Pieces Loosely Joined: A Unified Theory Of The Web''. Having this
mentality when developing the software will hopefully produce an application
that is easier to maintain.

The specific requirements for this project's framework and the things that
were considered when choosing it are discussed when looking at the development
of the application, in Section~\ref{sec:webframeworks}.
