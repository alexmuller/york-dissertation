%!TEX root = ../Project.tex

\subsection{User testing}

User testing is undertaken during the design and development of an
application. It can take many forms, but typically end users are interviewed
to gather their views and insights to help inform an application's design, or
alternatively users are observed using a prototype or other non-final version
of the application to gather feedback on the current design and locate and fix
any issues in the design.

Jakob Nielsen is a web usability expert who has been publishing articles on
his website, \url{http://www.useit.com/}, since 1995. In \emph{Why You Only
Need to Test with 5 Users} \cite{nielsen2000fiveusers}, Nielsen asserts that
usability tests should be run for all web projects, no matter how short the
project timescale or limited the budget. This is especially relevant for a
project such as this module allocation system, where the entire application
must be developed in under six months and there is no budget allocated.
Nielsen's advice is to run a single usability test with no more than five
volunteers, and to run different tests if more participants can be recruited.
His reasoning is that iterative design with testing after each iteration will
uncover any problems unwittingly created during the development process.
Finally, Nielsen points out that distinct groups of users need to be treated
separately during user testing. Nielsen's advice is sound, as five user tests
will not take a particularly long time to run and will cost nothing if
recruiting friends or colleagues -- but the payoff from simply having fresh
pairs of eyes looking at the software will be huge.

Nielsen Norman Group, a company founded by Nielsen with Don Norman in 1998,
publishes reports on web usability. Among the 230 tips offered in one such
report \cite{nng2001tipsusability}, Molich describes how to conduct user
testing sessions. The bulk of his recommendations are around making the test
participant feel comfortable during the session; this involves reassuring them
that it is not they who are being tested but the software, telling them that
they should simply perform the tasks as though they were at home and making
the first task simple to allow the participant to gain confidence before
moving on to operations that might be more complex.

Cennydd Bowles and James Box work for a web design agency based in Brighton.
In \emph{Undercover User Experience Design} \cite{bowles2011undercover},
Bowles and Box describe various methods of usability testing that can be
undertaken with little time or budget. They give advice on asking questions in
an unbiased way, so as not to influence the test. Like Nielsen, they advocate
around five user tests, stating that even one is better than none. The authors
suggest recording video (or, failing that, audio) of the interview as there
will not be enough time to take notes during the session.

Bowles and Box put forward another method of eliciting information from users,
namely the corridor test. This involves watching people use the current system
for a very short amount of time and observing any usability issues they
encounter. The primary advantage of this type of test is that it takes very
little time or effort on both the part of the participant and the researcher.

In a 1982 paper titled \emph{Pitfalls of user research, and some neglected
areas} \cite{brittain1982pitfalls}, J. M. Brittain sets out the different
kinds of study that can be carried out during the research phase of any
project. These are publishing a questionnaire or interviewing users, asking
users for any input they have regarding a system or service, and observing
users while they perform a task. One point made by Brittain is that user
research is occasionally too narrow-focused -- in his example, the library was
focusing ``upon the demands users make for documents'' without necessarily
considering how users read the documents once they are in possession of them.
In the case of the web, one could argue that user research focuses too much on
the specific task of interest and not on how users browse the web from
day-to-day or what they generally use the web for.

User research (interviews with potential users or the distribution of
questionnaires) should take place before the design phase in order to create a
desirable product. Kuniavsky \cite{kuniavsky2003userexperience} describes a
``family and friends usability test'' that he claims provides immediate
feedback on a prototype with minimal preparation and time required from the
research participant.

His key points for conducting a usability study are:

\begin{enumerate}
  \item Define your application's audience and the goals they want to achieve
  \item Create scenarios that will help them accomplish their goals
  \item Find test participants
  \item Observe them while they play out the scenarios defined
\end{enumerate}

Kuniavsky reinforces Molich's point that the most important consideration
during a usability test is that the participant is comfortable; for example,
that they understand it is not they who are being tested, but the application.
He says that participants should be strongly encouraged to narrate their
thought process as they use the application, as this provides useful insight
to the researcher when they review the studies later.

The recurring theme throughout all of these papers and books is that user
testing is absolutely key to creating an application that users will enjoy
interacting with. As the module allocation software hinges on students being
able to select their modules easily and quickly, there is no excuse to not
perform at least a small amount of user testing.
