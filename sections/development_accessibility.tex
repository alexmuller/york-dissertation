%!TEX root = ../Project.tex

\subsection{Accessibility}

Educational institutions such as the University of York are required by the
\gls{senda} 2001 to ensure that web content is accessible to disabled users.
The Web Office provides advice to those publishing web sites or applications
on behalf of the University at
\url{http://www.york.ac.uk/communications/websites/content/accessibility/}.
While this advice is more applicable to content-heavy pages such as department
websites for recruitment of undergraduate students, it still contains some
important things to consider while building a web application.

The two major accessibility issues surrounding a web application are users who
are visually impaired and users who are unable to use a traditional
interface-manipulation devices (a keyboard and mouse) to browse the web. To
mitigate the first accessibility issue, the application was tested to ensure
that users can increase the page or font size to fit their needs. Users who
use custom style sheets in their web browser (for example to enhance contrast)
are able to do so. All text and background colour combinations were tested to
ensure compliance with WCAG AAA, using a tool freely available online at
\url{http://www.snook.ca/technical/colour_contrast/colour.html}.

As to the second accessibility issue, generally speaking, I believe that users
who are unable to use a keyboard or mouse do not use JavaScript in their web
browser. The application is designed so that users with JavaScript disabled
are instead presented a simple form layout that can be read out to them by
screen-reading software and filled in solely with a keyboard (which can be
voice controlled).

It is thought with these considerations, every student will be able to use the
system to select their modules. However, even if the above fallbacks fail the
student will still be able to obtain help from their department (which could
complete the module selection on their behalf). The aim is that no student
will be stranded with an application they cannot use and no instructions
telling them how to gain assistance.

The student interface was tested by David Swallow, a Research Associate in the
\gls{hci} group at the University of York. Swallow commented that ``on the
whole, the accessibility of the application is excellent and passes many of
the checkpoints where other applications slip up''. He did, however, note
several minor issues with the application. The biggest such issue was with my
assumption that all users who do not have access to a mouse have JavaScript
disabled -- Swallow said that this fairly common assumption ``might have
stemmed from \gls{wcag} 1.0'', published in 1999. Regrettably there was not
sufficient time to research and mitigate this issue, but owing to the small
proportion of users who do not use a pointing device, I decided it was best to
simply note this as future work that should be undertaken.
