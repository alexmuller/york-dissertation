%!TEX root = ../Project.tex

\subsection{Online security}
\label{sec:research_security}

Recently there has been increased public awareness of online security because
of exploits orchestrated by groups such as ``LulzSec''. These groups have
exploited vulnerabilities in software and systems as well as through social
engineering, and have released sensitive information from large corporations
and governments \cite{ASATrendsinDataBreaches_2012}.

While some of their attacks have involved social engineering, those are
outside the scope of this report. The research conducted here focuses solely
on the technical methods that should be used to prevent unwanted disclosure or
destruction of information.

The \gls{owasp} Top 10 is a list of the ten most important security
vulnerabilities, as identified by The \gls{owasp} Foundation
\cite{OWASPTop10_2010}. These vulnerabilities range in impact, prevalence and
how sophisticated an attacker must be to exploit them.

The ten vulnerabilities provide a very solid grounding in security risks for
web applications. The list is updated frequently by \gls{owasp} (every three
years) in order to stay relevant. It is worth noting that as with all aspects
of information security, no system can ever be deemed to be perfectly safe.
The Top 10 simply aims to educate developers about the most prevalent risks
for web applications.

Not all of the ten security vulnerabilities are relevant to this project --
for example, some of them refer to the way in which the data is stored on
disk. The most relevant vulnerabilities are discussed in detail here.

\noindent{\textbf{Injection (A1)}}

Injection involves a malicious user entering specially-crafted input in order
to compromise an application. A specific example that would affect this
application is \gls{sql} injection, where by the input is crafted so that it
compromises or destroys the database. A request could be made like so:

\begin{lstlisting}
http://example.com/modules/?id=' OR '1'='1
\end{lstlisting}

In an insecure application, the string after \texttt{id=} would be entered
directly into a \gls{sql} statement:

\begin{lstlisting}
SELECT * FROM table WHERE id='' OR '1'='1';
\end{lstlisting}

This statement would then select everything from the database (as 1 always
equals 1), potentially returning data to the user that they should not be able
see. Similarly, it would be trivial to execute a \texttt{DROP TABLE} command
to destroy the database. The issue is mitigated by sanitising all user input
so that it is not executed directly in the database.

\noindent{\textbf{Cross-Site Scripting (A2)}}

Cross-site scripting (abbreviated XSS) is an attack involving a malicious user
sending data to the application which is then displayed without first being
sanitised. For example, in this application, an authenticated administrator
could upload a module with the name:

\begin{lstlisting}
<script>document.location='http://www.evil.example/exploit.html'</script>
\end{lstlisting}

If this module name was to be displayed to a student before being sanitised,
they would be automatically redirected to the malicious website, which could
be designed to steal their personal information.

\noindent{\textbf{Broken Authentication and Session Management (A3)}}

This entry in the Top 10 is more broad than the others, and I believe it is
designed to convey to developers the sheer number of ways in which
authentication systems can be broken. The recommendation given is to only use
authentication controls that are widely available - if a developer was to
write their own authentication mechanism, it would not be subjected to as much
testing as widely-known security packages and flaws would not be found as
easily.

\noindent{\textbf{Cross-Site Request Forgery (A5)}}

A cross-site request forgery (CSRF) attack occurs when a malicious user crafts
an \gls{http} request on their site (for example through adding a
\texttt{<form>} element) and then attempts to trick a legitimate user into
making the request. This particular attack can be mitigated through the
addition of an automatically-generated token to every input form or even every
page on the vulnerable site. This token can be checked on submission to ensure
that the user submitting the form came from the correct page, and therefore
that the request is legitimate.

\noindent{\textbf{Failure to Restrict URL Access (A8)}}

As \gls{http} is a stateless protocol, the user must be authenticated with
every request. A simple way to ensure this is carried out correctly is to keep
all requests relating to different access levels entirely separate. For
example, all basic user functions could be under \texttt{/modules} while all
departmental administrative actions could be under \texttt{/admin}. A global
security configuration would then ensure that a user had the appropriate
privileges for each request under those URLs.

\noindent{\textbf{Insufficient Transport Layer Protection (A9)}}

Any web application should be served as securely as possible - this is
especially important for authentication-related items such as cookies (a
cookie is a small amount of text data stored in a web browser and transmitted
to a web server). If a cookie is used with HTTP's stateless nature to
authenticate the user on every request, this cookie can be intercepted simply
by sniffing network traffic. Any user can then perform any action with the
privilege granted by this cookie.

The issue of cookie-sniffing over insecure connections has been known for many
years, but was brought to light in 2010 through the release of a Firefox
extension called ``Firesheep'', which provided a clear interface to allow any
user to steal credentials for popular sites (such as Amazon or Facebook) on an
insecure network \cite{FiresheepHow_2011}.
