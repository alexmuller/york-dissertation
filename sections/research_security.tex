%!TEX root = ../Project.tex

\subsection{Online security}
\label{sec:research_security}

Recently there has been increased public awareness of online security because
of exploits from groups such as ``LulzSec''. Groups such as this have
exploited vulnerabilities and released sensitive information from
organisations such as Sony, the FBI, etc [@todo reference required here].

While some of their attacks have involved social engineering, those is outside
the scope of this report. The research conducted here focuses solely on the
technical methods that should be used to prevent unwanted disclosure of
information.

The \gls{owasp} Top 10 is a list of the ten most important security
vulnerabilities, as identified by The \gls{owasp} Foundation
\cite{OWASPTop10_2010}. These vulnerabilities range in impact, prevalence and
how sophisticated an attacker must be to exploit them.

Not all of the ten security vulnerabilities are relevant to this project --
for example, some of them refer to the way in which the data is stored on
disk, which is out of scope for this software. The most relevant
vulnerabilities are discussed here.

\noindent{\textbf{Injection (A1)}}

Injection refers to a malicious user entering specially-crafted input in order
to compromise an application. A specific example that would affect this
application is SQL injection, where by input can be crafted so that it
compromises or destroys the database like so:

\texttt{http://example.com/modules/?id=' or '1'='1}

In this example, the first single quote...

\noindent{\textbf{Cross-Site Scripting (A2)}}

xxx

\noindent{\textbf{Broken Authentication and Session Management (A3)}}

xxx

\noindent{\textbf{Cross-Site Request Forgery (A5)}}

xxx

\noindent{\textbf{Failure to Restrict URL Access (A8)}}

xxx

\noindent{\textbf{Insufficient Transport Layer Protection (A9)}}

xxx
