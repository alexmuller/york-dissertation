%!TEX root = ../Project.tex

\subsection{Pilot by the Archaeology and History departments}
\label{sec:developmentpilot}

During the spring term (January to March 2012), the Archaeology and History
departments at the University of York used this application to allocate
modules to their second and third-year students. The application documentation
(Appendix~\ref{sec:documentation}) includes information for departmental
administrators about the setup of the application and limitations they should
be aware of.

A production instance of the application \gls{vm} was provided by
\gls{itservices} from 22 February, and the application was promoted from
\texttt{wwwtest.york.ac.uk} to \texttt{www.york.ac.uk} later that day after
final sign-off from the University's Information Security Officer. The
application was set up by departmental administrators on 23 February and the
setup was confirmed after a project group meeting that day.

Table~\ref{development_pilot_department_numbers} gives the amount of
information added to the system by the pilot departments.

\begin{table}
  \begin{center}
    \begin{tabular}{ | l | l | l | l | }
      \hline
      \textbf{Department}  & \textbf{Types of student (by degree)} & \textbf{Modules} & \textbf{Students} \\
      \hline
      History     & 14                           & 57      & 591      \\
      Archaeology & 8                            & 32      & 204      \\
      \hline
    \end{tabular}
  \end{center}
  \caption{Amount of data loaded into the application by the pilot departments}
  \label{development_pilot_department_numbers}
\end{table}

As the application was hosted by \gls{itservices} on the \texttt{york.ac.uk}
domain, notification was given to the team that provides end-user support.
Issues are logged by this team from a variety of sources, including emails to
\texttt{itsupport@york.ac.uk}, telephone calls and visits to the IT Support
office. During the trial period issues relating to this application were
forwarded to departmental administrators and the \gls{itservices} developer
assisting with the application.

The application was available to students from 0900 on Monday 27 February
until 1700 on Monday 5 March. % without interruption?

% The selection process was praised by students in the Archaeology department.
% A departmental administrator in Archaeology received the following comments:

Only one issue was raised by a user to their department during the week. The
user reported they were unable to use the drag-and-drop functionality to make
their choices. It was revealed that the user was using Internet Explorer 7, a
web browser that was superseded by Internet Explorer 8 in March 2009. As of
the writing of this report, computers provided on campus at the University of
York run IE8, as well as alternate web browsers such as Firefox and Opera (all
of which were tested for this application).

The project implementer had intended to include some text for users of old
browsers, indicating that the application had not been tested for them and
they should use an alternate browser if possible. Sites such as
\url{http://www.ie6nomore.com/} include snippets of code that demonstrate how
to achieve this. Unfortunately, for some reason this code was not committed to
the repository and did not make it into the release version of the software.

The student was advised by the departmental administrator to use a more
up-to-date web browser and successfully used a campus PC to select their
modules within 24 hours of the problem initially being raised.

Several students who were on study aboard programs reported issues connecting
to the University's \gls{vpn}, though these issues are out of scope for this
project. The issues could not be remedied quickly by \gls{itservices} due to
the lack of information available, and users experiencing problems sent their
choices by email. The choices were input into the system by administrators
acting on behalf of students. It is worth noting that if this were an
application built and supported by the University (as it will be if it is
available in future years), it would be available away from campus and these
\gls{vpn} issues would not have arisen.

On the afternoon of Friday 2 March the software was upgraded to include
improvements made during the week. Clearly performing a software upgrade in
the afternoon before a weekend is not ideal -- if there had been any issues,
they would not have been fixed quickly. However, the \gls{itservices}
developer responsible for helping with the application was away from the
office until 7 March. I decided that the upgrade had to be performed before
then in order for departments to attempt to make an allocation sooner rather
than later. If any issues had arisen during the upgrade, the developer would
have reverted to the previous working release version.

On the morning of Monday 5 March the History department reported they had
received emails from approximately ten students who had encountered login
errors when trying to use the application. \gls{itservices} noted that the
University had experienced problems with its authentication systems over the
weekend. Departments were then informed this was not an issue with the module
allocation application and should at that point be resolved.

A History administrator noted that by 10am on Monday 5 March (96\% of the way
through the application's opening time) just 71\% of students had used the
system to make their choices. However the department was unconcerned,
commenting that with time-limited processes such as this students tended to
wait until just before the deadline.

% During the pilot week, several students requested an enhancement to the
% interface. They requested that when making a choice for the second time, the
% ranking interface (Figure?) should not look as though it has never been used
% before -- it should load the choices made the previous time. This was not
% discovered during user testing as students were not given the opportunity to
% use the system twice.

In all, only one student raised an issue that prevented them using the
application during the week, which is equivalent to approximately 0.13\% of
all users. I would declare this a fairly successful trial of the student
interface.

The History department performed an allocation on the evening of 5 March,
which completed with Gurobi reporting an ``optimal'' status. Unfortunately, at
midday on 6 March the system stopped working, giving an ``Application Error''
to end users (this means an exception is being thrown in the Java code). On
investigation with \gls{itservices}, the database contained one less module
availability than module (as mentioned in
Section~\ref{developmentdatabasestructure}, these are separate entities). This
removal was either caused by a bug in the application code that was impossible
to reproduce during testing and has not since reappeared, or by user error
(accidental deletion) -- unfortunately, it is impossible to say which. The
removal of that entity resulted in the removal of all the student ranks
associated with that module, and so the database had to be restored to the
nightly backup made by \gls{itservices} on 6 March at 3am. This restoration
occurred on 9 March and the History department were able to perform another
allocation. @todo what happened next?

The Archaeology department performed an allocation on 6 March. The allocation
resulted in one module not being run (the solver found an optimal allocation
without that module), and the decision to discontinue the module had to be
referred to the department's Board of Studies (chaired by \gls{sa}). The
allocation of students to modules created by Gurobi was described as ``a
triumph'' by the department, and they were able to pass a file containing the
allocations to \gls{sits} one week before the deadline.

% @todo: what was the students perception of equity?

% @todo: how did the import go?
