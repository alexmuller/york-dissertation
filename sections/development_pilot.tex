%!TEX root = ../Project.tex

\subsection{Pilot by the Archaeology and History departments}

During the spring term (January to March 2012), the Archaeology and History
departments at the University of York used this application to allocate
modules to their second and third-year students. The application documentation
(Appendix~\ref{sec:documentation}) includes information for departmental
administrators about the setup of the application and limitations they should
be aware of.

A production instance of the application \gls{vm} was provided by
\gls{itservices} from 22 February, and the application was promoted from
\texttt{wwwtest.york.ac.uk} to \texttt{www.york.ac.uk} later that day after
final sign-off from the University's Information Security Officer. The
application was set up by departmental administrators on 23 February and the
setup was confirmed after a project group meeting that day.

Figure~\ref{development_pilot_department_numbers} gives the amount of
information added to the system by the pilot departments.

\begin{table}
  \begin{center}
    \begin{tabular}{ | l | l | l | l | }
      \hline
      \textbf{Department}  & \textbf{Types of student (by degree)} & \textbf{Modules} & \textbf{Students} \\
      \hline
      History     & 14                           & 57      & 591      \\
      Archaeology & 8                            & 32      & 204      \\
      \hline
    \end{tabular}
  \end{center}
  \caption{Amount of data loaded into the application by the pilot departments}
  \label{development_pilot_department_numbers}
\end{table}

As the application was hosted by \gls{itservices} on the \texttt{york.ac.uk}
domain, notification was given to the team that provides end-user support.
Issues are logged by this team from a variety of sources, including emails to
\texttt{itsupport@york.ac.uk}, telephone calls and visits to the IT Support
office. During the trial period issues relating to this application were
forwarded to departmental administrators and the \gls{itservices} developer
assisting with the application.

The application was available to students from 0900 on Monday 27 February
until 1700 on Monday 5 March. % without interruption?

% The selection process was praised by students in the Archaeology department.
% A departmental administrator in Archaeology received the following comments:

% @todo: comments from the week

% @todo: provide details of issues raised during the week

Only one issue was raised by a user to their department during the week. The
user reported they were unable to use the drag-and-drop functionality to make
their choices. It was revealed that the user was using Internet Explorer 7, a
web browser that was superseded by Internet Explorer 8 in March 2009. As of
the writing of this report, computers provided on campus at the University of
York run IE8, as well as alternate web browsers such as Firefox and Opera (all
of which were tested for this application).

The project implementer had intended to include some text for users of old
browsers, indicating that the application had not been tested for them and
they should use an alternate browser if possible. Sites such as
\url{http://www.ie6nomore.com/} include snippets of code that demonstrate how
to achieve this. Unfortunately, for some reason this code was not committed to
the repository and did not make it into the release version of the software.

The student... % had their choices made by an administrator acting on their behalf?

Only one student raised an issue during the week, which is equivalent to
approximately 0.13\% of all users.

On [@todo 5ish] March the software was upgraded to include alterations made
during the week, and on [@todo 6ish] March an allocation was performed by each
of the departmental administrators.

% @todo: include info on how the allocation goes
