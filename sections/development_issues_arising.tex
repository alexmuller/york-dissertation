%!TEX root = ../Project.tex

\subsection{Issues arising during implementation}
\label{sec:issuesarising}

% Discuss some of the _real_ issues of a _real_ project

\noindent{\textbf{Availability of stakeholders}}\mynobreakpar

One of the most positive aspects of undertaking this project was in fact one
of the most difficult -- that is, the project was designed to work very
closely with the University of York and as such relied on ``real''
stakeholders.

An issue arose fairly early in the project when it became clear that some
administrative staff (both IT support staff and those in the pilot
departments) would be away from work for prolonged periods of time or would
leaving their jobs during the creation of this application. This caused delays
while the replacement staff had to be brought up to speed on the background
and current state of the project.

The issue was mitigated with the help of the \gls{aso}, which was responsible
for assisting with issues not directly related to implementation. The new or
replacement staff were then invited to group meetings so that they could keep
up to speed with recent developments and provide any necessary input. As
implementation began to finish, departmental administrators were provided
training on the software by the project implementer. A deputy was present at
the training sessions so that the software could still be used in case of
unforeseen circumstances rendering the departmental administrator unable to
set up the software.

\noindent{\textbf{Lack of technical contacts available during the project}}\mynobreakpar

I first met the Head of Web Services and an \gls{itservices} developer on 16
January, several months after the project had begun, and this meant that the
amount of time allocated to implementation was reduced. If a project of this
nature was to be carried out in future, the technical contacts
(\gls{itservices} in the case of the University of York) should be involved as
early as possible. I am of the opinion that earlier involvement from a
developer would have resulted in the developer actually spending less time
assisting with the project overall.

\noindent{\textbf{Inability to interact with the system while it contained sensitive data}}\mynobreakpar

The project implementer was unable to respond to bugs in a timely manner as it
was not possible for \gls{itservices} to provide access to the database and
logs for an application containing live student data. This made bug fixing in
the live environment nearly impossible.

Specific bugs that occurred during the pilot are described later, in
Section~\ref{sec:developmentpilot}. Generally, the process surrounding bug fixing
was as follows:

\begin{enumerate}
  \item An error is reported by a departmental administrator
  \item The administrator contacts \gls{itservices} and/or the project implementer
  \item They liaise (in order to provide redacted log files, if necessary)
  \item The software is patched by the project implementer
  \item The patched software is deployed by \gls{itservices}
\end{enumerate}

Because of the dependency between the various different people involved in the
project, fixing a bug was a process that took several hours. While releasing a
patched application in one or two days is relatively uncommon in, for example,
desktop software development, it is a long time for a web application that is
only in use for two weeks.

The web application was only available to students for a week, or about 120
waking hours. Under the assumption that the use of the application by the 800
students was distributed uniformly over that time, a several hour period during
which the software was not fully functional might inconvenience approximately
forty students.

The project implementer signed a non-disclosure agreement in order to protect
the student data involved in the system, but ultimately it was still felt by
the University that a student could not be given access to other students'
personal data.

\noindent{\textbf{Unfamiliarity with the language and framework}}\mynobreakpar

While the framework was chosen based on its maintainability as
University-owned software, the project implementer was not particularly
familiar with the underlying language and framework. This made it hard to
estimate the amount of time that various stages of implementation would take.
If this project was to be undertaken again, the implementer should first use a
language they are familiar with to prototype the application quickly, and then
use any remaining time to re-implement in a language that can be maintained by
the future software owners.

This would require almost double the implementation effort (though parts, such
as the user experience and some front-end code, could be reused across
frameworks) but should allow the implementer to provide clients with a working
application far in advance of any deadlines. When developing using unfamiliar
software, the developer is dependent on how fast they are able to learn and
use new techniques.
