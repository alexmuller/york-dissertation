%!TEX root = ../Project.tex

\subsection{Maintainability and the future of the software}
\label{sec:maintainability}

As the software created during this project will have to be maintained by the
University's \gls{itservices} if it is evaluated as successful, great care
must be taken to ensure that the application is implemented in the most
extensible and maintainable way possible.

Principles of good software engineering apply equally to web applications as
to any other software project. There are several basic methods recommended by
Green and Ledgard \cite{CodingGuidelines_2011} for writing readable and
maintainable code. Their recommendations are written with the aim that the
eventual maintainer of the software will have to do as little work as possible
to understand the purpose and implementation of a given piece of code. Some of
the more applicable recommendations are:

\begin{itemize}
  \item Align parts of the code (e.g. equal signs)
        vertically when it makes sense to
  \item Write lines no longer than approximately 70 characters,
        and use line breaks if necessary
  \item Use simple English and short names for things that will
        be referenced frequently
  \item Add blank space around operators (e.g. \texttt{3 + 2 = 5}
        rather than \texttt{3+2=5})
  \item Indent \texttt{if} statements to allow the reader to scan
        the code more easily
  \item Comment code when necessary, but comments are not a
        solution for bad code
\end{itemize}

They note that for any system that may have a long lifespan, every effort
should be taken to improve ``readability and maintainability''. A project that
will only be used for a short amount of time by the original author does not
require as much attention to maintainability as one which will last several
years and be maintained by several different programmers.

One of the big concerns when considering maintainability is the environment
that the software will be maintained in and who will be responsible for it.
This application will be maintained by a group in \gls{itservices} called the
Web Services Group, which consists mostly of Java developers. Implementing the
software in a language they are unfamiliar with or unable to host easily would
require them to spend far more time working on the application and would
dramatically reduce the value it would provide.
