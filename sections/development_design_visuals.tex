%!TEX root = ../Project.tex

\subsection{Design decisions and visual appearance}

The application should be visually consistent with other University web
software to instill trust in the user. This means placing the logo in the
correct position and at the correct size, using University colours and the
standard header and footer as far as possible. The University's Web Office
(part of the Communications Office) was contacted during the project to
enquire as to whether the application should be branded using the standard
green header, or as a part of the student portal which is coloured grey and
blue (Figure~\ref{yorkacuk_student_portal}). The Web Office felt that the
application should initially be branded using the green colour scheme, but
that if it was more tightly integrated with the student portal in future years
then it would make sense to adjust the design to reflect that.

One area that usability gains can be made are in relation to how the
application saves data. During one of the user research sessions, the
participant explained that while choosing optional modules during her first
year, she neglected to press that application's ``OK'' button and, unaware
that she had not submitted choices, was unable to take her preferred modules.
As an explicit ``Save'' button should be retained to reduce surprise and allow
the user to feel in control, the most sensible decision is to automatically
save user input periodically and notify users that their choices are being
saved. The submit button would save data in case of failure of the automatic
save mechanism.

However, there is a balance to be struck here. It was felt by the pilot
departments that it would be simpler if the application only included a submit
button and did not attempt to automatically save choices. This additional
feature would have introduced complexity into the system, and therefore could
have caused unforeseen problems.
