%!TEX root = ../Project.tex

\subsection{Development methodology}

The methodology used during a software development project influences how the
development of an application progresses over time. It can define at which
stage of the process prototyping, planning, development and evaluation occur.

% Prototyping

Zhang and Chung \cite{MODFM_2003} note that prototyping can be used to
reinforce client confidence as well as making better use of the time allocated
to development and implementation. Bochicchio and Paiano
\cite{PrototypingWebApplications_2000} note that creating prototypes of web
applications has several advantages, the most relevant of which for this
project is that a mockup can be used to get feedback from non-technical
stakeholders such as the project manager and the departmental contacts. This
makes sense, as providing a visual aid can only benefit the project clients
when they are trying to articulate their thoughts and comments.

Prototypes can be defined as being low or high fidelity depending on how much
detail is included and how closely they are designed to resemble the final
application. The lowest fidelity prototypes can be creating using a marker pen
and sheets of blank paper, whereas higher fidelity prototypes could be written
in \gls{html} to appear in a web browser and allow the user to interact as
they might with the final system.

% Iterative

Iterative development (also sometimes referred to as \gls{rad}) is a process
by which the an initial product is gradually improved through trial and error.
This differs from processes such as the waterfall method, where testing
follows implementation, which follows design, which follows gathering of
requirements. The key advantage of following an iterative development process
is that it allows far more flexibility than other methodologies; it is common
in software development that the requirements may change or be refined over
the duration of the project, and the development process should be able to
adapt as necessary \cite{kuniavsky2003userexperience}.

Kuniavsky points out that an iterative development process is especially
suitable for web applications, as prototypes can be created quickly. A low
fidelity prototype of a web application (which might consist of sketched
wireframes) can be created in a matter of minutes, while slightly higher
fidelity prototypes such as a simplified application front-end can be running
in a web browser within a day.

An iterative development process for this project might involve background
research with users (in the form of interviews), the creation of a prototype,
refining the user experience through more interviews, and repeating the
``prototype $\rightarrow$ interview $\rightarrow$ refine user experience
$\rightarrow$ interview'' cycle. As the requirements for this project are
initially fairly loosely defined, an iterative process would make sense as it
should be able to adapt best to any late changes in requirements.
