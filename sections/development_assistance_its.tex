%!TEX root = ../Project.tex

\subsection{Assistance provided with the application by IT Services}

\subsubsection{Development work}

An \gls{itservices} developer was provided to assist with the integration of
the project into the University's infrastructure. As the framework was chosen
so that it was familiar to the University's developers, a certain amount of
code reuse could occur between this system and other already existing
projects.

The code reuse was facilitated by this system using the same authentication
mechanism as the student portal (Section~\ref{sec:webapps_york}), another Java
application. The code to authenticate students and staff (using Spring
Security) was included into the module allocation application by the
\gls{itservices} developer.

Code for this application was committed using a \gls{dvcs} (Git) to \gls{github},
which makes it easy to see which commits came from which author.

\gls{github} provides an interface to show how much code came from each
developer. On \gls{github}, this figure is given in terms of ``impact'' (the
total number of lines of code modified, which is equivalent to lines added and
deleted). \gls{github} reports that the \gls{itservices} developer modified
336 lines of code. \mbox{``SLOCCount''} \cite{SLOCCount} (a tool for
generating statistics on source code) reports that the entire project has
2,026 Total Physical Source Lines of Code.

\subsubsection{Infrastructure}

\gls{itservices} provided a \gls{vm} for testing as well as another to host a
production instance of the application. They also provided an Oracle database.
The software was developed using a MySQL database, but as it used an
\gls{orm}, this simply required a minor configuration change so that the
application knew where to find the production database.

Finally, once the software had been deployed to the production instance, they
mapped a URL from under \texttt{www.york.ac.uk/students/} to connect students
and staff to the application.

\subsubsection{Security review}
\label{sec:securityreview}

On 15 February 2012, a review of the application architecture was conducted by
the University's Information Security Officer and the developer mentioned
previously who assisted with integrating the application's security
configuration.

As well as being targeted at departments, the software documentation created
by the project author also describes possible security vulnerabilities and how
they were mitigated. This documentation is provided in
Appendix~\ref{sec:documentation} and was given to those present at the
security review meeting.

The web hosting configuration and application security model was reviewed. Two
recommendations were made about the code; primarily, that the application
should use the \gls{owasp} ESAPI library for escaping special characters to
prevent some attacks noted in Section~\ref{sec:research_security}.

The security expert considered that the data held by the system could be
classed as between medium and low risk, as there was no highly sensitive
personal information such as student telephone numbers or email addresses.
