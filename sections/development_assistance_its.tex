%!TEX root = ../Project.tex

\subsection{Assistance provided with the application by IT Services}

\subsubsection{Development work}

An \gls{itservices} developer set aside some time to assist with the
integration of the project into the University's infrastructure. As the
framework was chosen so that it was familiar to the University's developers, a
certain amount of code reuse could occur between this system and other already
existing projects. The code reuse was facilitated by this system using the
same authentication mechanism as the student portal
(Section~\ref{sec:webapps_york}), a Java application that the developer
helping with the project was responsible for implementing. The code to
authenticate students and staff (using Spring Security) was included into the
module allocation application by the developer.

Code for this application is stored using a \gls{dvcs} (specifically Git) to
\gls{github}, which makes it easy to see which commits (and therefore which
work) came from which author. \gls{github} provides an interface to show how
much code came from each developer. On \gls{github}, this figure is given in
terms of ``impact'' (the total number of lines of code modified, which is
equivalent to lines added and deleted). \gls{github} reports that the
\gls{itservices} developer modified 336 lines of code. \mbox{``SLOCCount''}
\cite{SLOCCount} (a tool for generating statistics on source code) reports
that the entire project has 2,023 Total Physical Source Lines of Code.

\subsubsection{Infrastructure}

\gls{itservices} provided a \gls{vm} for testing and development as well as
another to host a production instance of the application. They also provided
an Oracle database. The software was developed locally using a MySQL database,
but as it used an \gls{orm}, this simply required a minor configuration change
so that the application knew how to connect to the production database. Once
the software had been deployed to the production instance, they mapped a URL
from under \texttt{www.york.ac.uk/students/} to allow students and staff to
access the application.

\subsubsection{Security review}
\label{sec:securityreview}

On 15 February 2012, a review of the application architecture was conducted by
the University's Information Security Officer, the Head of Web Services and
the developer mentioned previously who assisted with integrating the
application's security configuration. The software documentation describes
possible security vulnerabilities and how they were mitigated. This
documentation is provided in Appendix~\ref{sec:documentation} and was given to
those present at the security review meeting.

The web hosting configuration and application security model was reviewed. Two
recommendations were made about the code; primarily, that the application
should use the \gls{owasp} ESAPI library for escaping special characters to
prevent some attacks noted in Section~\ref{sec:research_security}. The
security expert was of the opinion that the data held by the system could be
classed as between medium and low risk, as there was no highly sensitive
personal information such as student telephone numbers or email addresses.
