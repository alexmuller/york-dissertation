%!TEX root = ../Project.tex

\section{Further work}
\label{sec:furtherwork}

% A chapter describing possible ways in which your work could be continued or
% developed. Be imaginative but realistic.

There were many feature requests from the project steering group that I had
insufficient time to implement during the course of the project. These
requests (including their difficulty and potential implementation strategy)
are discussed in this section.

\subsection{Maintaining a history}

While considering the data protection implications mentioned in
Section~\ref{sec:dataprotection}, a popular request from the pilot departments
was that the application should retain its knowledge about a student for the
duration of their time at university.

One obvious benefit of this improvement is that the fairness of the allocation
for a given student could be improved over the course of their academic
career. For example, if a student was given one of their lower ranked modules
in the first year, the allocation could compensate them by giving them
preference over other students for a higher ranked module in the second
year.

The University's Data Protection Officer would have to be consulted in order
to draft a suitable retention policy for this additional data. It seems that
this is data that a department might be expected to collect and reference, and
as such could be held in line with the department's retention policies.

Technically speaking, this is not a particularly difficult feature to
implement. Each student can be given an average score, equivalent to the sum
of the ranks they were allocated divided by the number of modules. Then,
students with a higher average score have a greater weight given to their top
choices in the objective function.

\subsection{Different methods of obtaining information from students}

One mockup created during the user testing allowed students to weight modules
rather than ranking them -- it was thought by the project steering group that
this interface (Figure~\ref{prototype_student_weighting}) would allow students
to express more clearly their preferences for certain modules (i.e. the
student could use the bars to demonstrate that they have a very strong desire
for module X, and absolutely do not want to be allocated module Y). However,
user testing revealed that students were confused about how the values shown
on the bars would be used to allocate modules. Furthermore, a concern raised
at both the project steering group meetings and by students during testing was
that this nuanced a selection process might evolve into a ``bidding war''
between students, with a student adding one or two points to the values their
peers had input in order to secure a module. In summary, there was strong
feedback from the test groups that the application should in fact be kept as
simple as possible -- the phrase \gls{kiss} is often used.

It is possible that an interface similar to the one prototyped could be
further refined such that students are comfortable using it -- the project
group and implementer still believe it could provide a ``better'' allocation
for students. A simple solution could perhaps be to not display the exact
percentage figures to students so that they are choosing modules based on the
approximate lengths of the bars rather than the exact values.

\subsection{Automatic import and export}
\label{sec:autoexport}

The application could be extended so that it automatically accesses data
stored in the University's Data Warehouse. This would be far easier for an
experienced University developer to implement than it would be for the project
author, who has no prior experience writing software to integrate with
University of York infrastructure.

This would remove the need for departments to set up the application by
uploading files and, crucially, would remove the requirement that \gls{ssdt}
must import the data into \gls{sits} when the allocation is complete. The
result of this integration would be a further reduction in administrative time
required.

\subsection{Expanding the use of the application}

\gls{sa} posed the question of whether this application could be used to allow
visiting students to select their modules. The application has been designed
in such a way that any course, no matter what the structure, can be added. The
only blocking factor for this use case is that the application requires a
University of York login to authenticate with Shibboleth. A standalone version
of the application could be created with its own authentication mechanism,
which would avoid this requirement.

\subsection{Contacting students}

During user testing, one benefit students noted compared to the paper system
is that it is significantly easier to keep a record of the choices they made.
Currently, the system simply advises students to print or save the page that
is displayed after they have made their choices. An obvious improvement to
this process is that students could be emailed a copy of their choices. This
would confirm to the student that their input has definitely been saved, while
also giving them a copy of their choices.

\subsection{Support for more complex course structures}

Several departments have complex course structures, for example where a module
in one term influences the module that must be taken in the next term. The
interface could be extended to reflect this.

\subsection{Improved application code and maintainability}

There are many places in which the application code could be improved. It is
currently quite fragile, and better error handling would dramatically improve
the user experience. Additionally, as different departments have their own
requirements for the algorithm, it would be an improvement if they could
manipulate small parameters in the algorithm themselves, such as the
coefficients used in the objective function.

\subsection{Other minor features}

There are countless other minor features that could be implemented. For
example, the system currently records a timestamp for every student when they
save their choices. It may be interesting (though is not at all required) for
the departments to be able to see at what point during the allocation period
the students are making their choices, so graphing this information may be
useful.

As noted in Section~\ref{sec:developmentpilot}, the student ranking interface
should also be improved so that on subsequent uses of the software by students
the ranking interface is already filled with the choices the student has made.
As the software is currently coded, a student might find it confusing when, on
performing a second ranking, they are greeted with an interface that appears
as though they have never used the software before.
