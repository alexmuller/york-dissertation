%!TEX root = ../Project.tex

\section{Further work}
\label{sec:furtherwork}

% A chapter describing possible ways in which your work could be continued or
% developed. Be imaginative but realistic.

There were many feature requests from the project steering group that I had
insufficient time to implement during the course of the project. These
requests (including their difficulty and potential implementation strategy)
are discussed in this section.

\subsection{Maintaining a history}

While considering the data protection implications mentioned in
Section~\ref{sec:dataprotection}, a popular request from the pilot departments
was that the application should retain its knowledge about a student for the
duration of their time at the University. One obvious benefit of this
improvement is that the fairness of the allocation for a given student could
be improved over the course of their academic career. For example, if a
student was given one of their lower ranked modules in the first year, the
allocation could compensate them by giving them preference over other students
for their first choice in the second year.

The University's Data Protection Officer would have to be consulted in order
to draft a suitable retention policy for this additional data. It seems that
this is data that a department might be expected to collect and reference in
the normal course of a student's career, and as such could be held in line
with the department's retention policies.

Technically speaking, this is not a particularly difficult feature to
implement. Each student can be given an average score, equivalent to the sum
of the ranks they were allocated divided by the number of modules. Then,
students with a higher average score have a greater weight given to their top
choices in the objective function. Some testing would have to be undertaken
with anonymised student choices and allocations in order to ensure that the
allocation being made was still, on the whole, fair.

\subsection{Different methods of obtaining information from students}

One mockup created during the user testing allowed students to weight their
modules rather than ranking them -- it was thought by the project steering
group that this interface (Figure~\ref{prototype_student_weighting}) would
allow students to express more clearly their preferences for certain modules
(i.e. the student could use the bars to demonstrate that they have a very
strong desire for module X, and absolutely do not want to be allocated module
Y). However, user testing revealed that students were confused about how the
values shown on the bars would be used to allocate modules. Furthermore, a
concern raised at both the project steering group meetings and by students
during testing was that a selection process this nuanced might evolve into a
``bidding war'' between students, with a particular student adding one or two
points to the values their peers had input in order to secure a module. In
summary, there was strong feedback from the test groups that the application
should in fact be kept as simple as possible -- the phrase \gls{kiss} is
sometimes used in software development. It is possible that an interface
similar to the one prototyped could be refined such that students are
comfortable using it -- the project group and implementer still believe it
could provide a ``better'' allocation for students, as the system would be
given more information on which to base the allocation. A simple solution
could perhaps be to not display the exact percentage figures to students so
that they are choosing modules based on the approximate lengths of the bars
rather than the exact values.

All aspects of the student interface must be tested in detail to ensure that
they are fully accessible to users who do not use a keyboard or mouse. These
tests should include researching possible methods for providing alternative
methods of input for users without traditional input devices.

\subsection{Automatic import and export}
\label{sec:autoexport}

The application could be extended so that it automatically accesses data
stored in the University's Data Warehouse, a centralised data store which
contains \gls{sits}. This would be far easier for an experienced University
developer to implement than for me, as I have no prior experience writing
software to integrate with the University of York's infrastructure. This
enhancement would remove the need for departments to set up the application by
uploading \gls{csv} files and, crucially, would remove the requirement that
\gls{ssdt} must import the data into \gls{sits} when the allocation is
complete. The result of this integration would be a further reduction in
administrative time required.

\subsection{Expanding the use of the application}

\gls{sa} posed the question of whether this application could be used to allow
visiting students to select their modules in future. The application has been
designed in such a way that any course, no matter what the structure, can be
added. The only blocking factor for this request is that the application
requires a University of York login to authenticate with Shibboleth. A
standalone version of the application could be created with its own
authentication mechanism, which would avoid this requirement.

\subsection{Contacting students}

During user testing, one benefit students noted compared to the paper system
is that it is significantly easier for them to keep a record of the choices
they made. Currently, the system simply advises students to print or save the
page that is displayed after they have made their choices. An improvement to
this process is that students could be emailed a copy of their choices on
submission. This would confirm to the student that their input has definitely
been saved, while also giving them a copy of their choices. Again, this
enhancement is easier for a developer who has experience sending email using
University of York infrastructure.

\subsection{Support for more complex course structures}

Several departments have complex course structures, for example where a module
in one term influences the module that must be taken in the next term. The
interface could be extended to reflect this. Some members of the pilot
departments mentioned that English is a department with more complex course
and module structure, so work should be undertaken with that department to
ensure the software is as flexible as it can be.

\subsection{Improved application code and maintainability}

As with any first version of a piece of software, there are many places in
which the application code could be improved. It is currently quite fragile,
and better error handling would dramatically improve the user experience.
Additionally, as different departments have their own requirements for the
algorithm, they would greatly appreciate being able to manipulate parameters
in the allocation code themselves, such as the coefficients used in the
objective function.

In terms of maintainability, there are several places that require
improvement. One particular example is that the coefficients used in the
objective function, if static, should not be hard-coded into the Java code
(line 43 of the file given in Appendix~\ref{sec:gurobicode}). As noted in the
previous paragraph, these coefficients should ideally be stored in the
database and should be able to be modified by the departments.

The application should undergo more rigorous testing in order to locate any
issues -- this could be done through writing more detailed unit tests than the
ones generated automatically by Spring Roo. Performance testing should be
conducted, especially if the number of students using the application in
future is significantly higher than during the pilot.

\subsection{Other minor features}

There are countless other minor features that could be implemented. For
example, the system currently records a timestamp for every student when they
save their choices. It may be interesting for the departments to be able to
see at what point during the allocation period the students are making their
choices, so graphing this information may be useful.

As noted in Section~\ref{sec:developmentpilot}, the student ranking interface
should also be improved so that on subsequent uses of the software by students
the ranking interface is already filled with the choices the student has made.
As the software is currently coded, a student might find it confusing when, on
performing a second ranking, they are greeted with an interface that appears
as though they have never used the software before.

Both departments commented that the files that they export could be better
structured. This reveals that more time should have been spent detailing the
administrative interface with departments. They requested that the \gls{csv}
files also contain module names as well as module codes, and that students
were ordered or grouped by year and by \gls{routecode}. It is worth noting
that none of these requests are technically difficult to implement, which
indicates that more time should have been spent with departments when
designing the administrative functionality.
