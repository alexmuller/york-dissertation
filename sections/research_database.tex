%!TEX root = ../Project.tex

\subsection{Database design}
\label{sec:researchdatabase}

A \gls{dbms} is software that looks after a data store and generally might
provide the ability to add or edit records in the database. The \gls{dbms}
will be one of the more mature products used in the creation of this system,
with many having been available since the 1990s. Relational databases are a
common feature of web applications. In meetings with \gls{itservices} during
the research stage, it was discovered that the University of York already
deploys MySQL and Oracle database systems for its web applications and
\gls{mis}. Database administrators at the University are familiar with these
two relational systems.

When designing a system, Johnson \cite{DatabaseModelsLanguagesDesign} gives
several questions that he believes must be answered before a database can be
created:

\begin{itemize}
  \item What are the entities that need to be stored by the database?
  \item What are the relationships between these entities?
  \item What constraints are there on the database?
  \item What kind of queries will be written against this database?
\end{itemize}

All the questions above are relevant during the design of the database. While
the first three questions relate to the structure of the database, the final
question is especially important when considering database performance. His
method involves drawing an \gls{er} diagram (as the name implies, this answers
the first two questions in the form of a graph) and then translating the
\gls{er} diagram into the database schema, which includes deciding which
fields will become primary and foreign keys. Any constraints (such as minimum
or maximum lengths of strings) can then be added depending on the \gls{dbms}
product being used. Additionally, Johnson says that as far as possible,
constraints on the entities should be enforced by the \gls{dbms} and not by
the application logic. Most importantly, this avoids duplication of effort in
the application's code and therefore reduces the likelihood of a bug.

Johnson gives an overview of functional dependency analysis, which he calls a
``complementary process'' to the creation of an \gls{er} diagram. He goes on
to describe the \gls{bcnf}, beginning with the statement that no data should
be duplicated in the database. Duplication of data in this project might
involve, for example, storing a department's name against every module that
department has. It makes far more sense for each module to simply store a link
to a single department entity, which holds details about that department.
Reducing duplication is important as it removes any possibility of
inconsistency in the data, as well as reducing the amount of data that must be
stored.

Normal forms are labels given to databases that can be used to describe the
likelihood or possibility of that database containing inconsistent or
incorrect data. A database in \gls{3nf} could be said to be in `better' shape
than one in \gls{1nf}, as it is less likely to harbour inconsistencies.
Johnson says that decomposing databases beyond \gls{bcnf} or \gls{3nf} may
have negative consequences such as requiring more complex queries where they
would not otherwise be needed.

Elmasri and Navathe give a clear overview of normal forms in
\emph{Fundamentals of Database Systems} \cite{ElmasriFundamentals_2004}.
\Gls{1nf} involves ensuring that each attribute only includes single values
and there should be no ``nested relations''. The example that they give is
that a department's `location' field should not contain a list of locations,
and that instead a composite primary key should be created. \Gls{2nf} ensures
that there is a full dependency on any multi-attribute key. \Gls{2nf} is
tested for by checking every primary key that contains more than one attribute
-- if it is possible to remove one attribute, the database is not in
\gls{2nf}. Finally, for a database to be in \gls{3nf} no attribute must be
dependent on another attribute apart from the primary key. This is referred to
as transitive dependency. The first definitions of normal forms (originally by
E. F. Codd in the 1970s) are paraphrased by Elmasri and Navanthe; one
important thing to note is that these forms are contained within each other
(that is, any database in \gls{3nf} is automatically \gls{2nf} and \gls{1nf}).
