%!TEX root = ../Project.tex

\subsection{Integration with other University software}

The system will be required to pull data from and push data to \gls{sits}, as
this is the centrally maintained database that always contains a true and
accurate representation of every student's data. Data stored in \gls{sits} is
used throughout the University, including student administration, timetabling
and room allocation, but is most important for the staff in the department
that the student is a member of. This system will interact with data stored in
\gls{sits} at two points:

\begin{enumerate}
  \item Module and student data must be imported (\gls{sits} $\rightarrow$
        module allocation application) during setup
  \item Module allocation data must be exported (module allocation application
        $\rightarrow$ \gls{sits}) after the allocation algorithm has run
\end{enumerate}

The import and setup will be performed by departmental administrators. This
decision was taken to reduce the amount of time required from \gls{ssdt},
especially as administrators will need to spend time tailoring the system
after the import in any case.

For the pilot of this module allocation software, the data produced by the
application will be processed manually by \gls{mh}. He will perform a ``sanity
check'' to ensure that the data generated by the application looks accurate
and will import it into \gls{sits}. If this application is evaluated
successfully and maintained centrally, it will not be feasible to import the
data manually for each department and some kind of automatic import will have
to be developed (discussed later in Section~\ref{sec:autoexport}).
