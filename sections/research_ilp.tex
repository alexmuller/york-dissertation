%!TEX root = ../Project.tex

\subsection{Integer linear programming solvers}

% [@todo talk about their use in industry?]

Allocating modules to students can be represented as an integer linear
programming problem. Every module that a student can possibly be allocated is
assigned a binary variable. Each of these variables is set by the solver to 0
if the allocation does not take place, or 1 if it does. For example, a system
containing two users and two modules would be represented as:

\begin{verbatim}
  user1_moduleA  0/1
  user1_moduleB  0/1
  user2_moduleA  0/1
  user2_moduleB  0/1
\end{verbatim}

We research \gls{lp} solvers in order to find one that can produce the
``best'' allocation of students to modules given several constraints, such as
the number of students in the module.

The basic requirements for a solver for this project are as follows, and are
explained in more detail below.

\begin{itemize}
  \item Interface in the language of the rest of the project
  \item Easy to understand and modify in future (maintainable)
  \item An appropriate software licensing policy
\end{itemize}

The solver should have an appropriate software interface so that it will
integrate easily with the rest of the application. This ensures that little
additional technical knowledge will be required to understand how the model
has been constructed.

Similarly, the solver should have a clear syntax for setting up, processing
and retrieving results from the model. As the software is likely to be
maintained by software developers who do not necessarily have experience with
this kind of constraint optimisation software, clear syntax will ensure the
software is as maintainable as possible.

Finally, the solver should have a license that is compatible with its expected
use (in an academic insitution).

% \subsubsection{SCIP}
% 
% SCIP
% 
% \subsubsection{Gurobi}
% 
% Gurobi
% 
% \subsubsection{MiniZinc}
% 
% MiniZinc?
