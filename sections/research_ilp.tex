%!TEX root = ../Project.tex

\subsection{Integer linear programming solvers}
\label{sec:researchilp}

% [@todo talk about their use in industry?]

Allocating modules to students can be represented as an integer linear
programming problem. Every module that a student can possibly be allocated is
assigned a binary variable. Each of these variables is set by the solver to 0
if the allocation does not take place, or 1 if it does. For example, a system
containing two users and two modules would be represented as:

\begin{verbatim}
 variable         allocation_made
----------------------------------
 user1_moduleA    0
 user1_moduleB    1
 user2_moduleA    1
 user2_moduleB    0
----------------------------------
\end{verbatim}

Using a solver instead of writing an algorithm from the ground up has numerous
advantages, the most important of which is that it will dramatically reduce
the time required for implementation. It will also increase flexibility, as
constraints can be added and removed easily. Existing solvers are commonly
used to solve optimisation problems that are orders of magnitude larger than
the thousands of variables (\emph{student, module} pairs) this application
will have to deal with.

I research \gls{lp} solvers in order to find one that is most suitable for
this project. The basic requirements for a solver for this project are
summarised and then explained in more detail below.

\begin{itemize}
  \item Has an interface in the language used in the rest of the project
  \item Is easy to understand and modify in future (maintainable)
  \item Has an appropriate software licensing policy
\end{itemize}

The solver should have an appropriate software interface so that it will
integrate easily with the rest of the application. This ensures that little
additional technical knowledge will be required to understand how the model
has been constructed.

Similarly, the solver should have a clear syntax for setting up, processing
and retrieving results from the model. As the software is likely to be
maintained by software developers who do not necessarily have experience with
this kind of constraint optimisation software, clear syntax will ensure the
software is as maintainable as possible.

Finally, the solver should have a license that is compatible with its expected
use (in centrally maintained software at an academic institution).

\textbf{SCIP (Solving Constraint Integer Programs)} is a non-commercial solver
that is available free for academic use. It is written in C and includes a C++
wrapper. The earliest release detailed on the SCIP site is version 0.80 in
late 2005, and the most recent is a bug fix release in December 2011.

\textbf{Gurobi Optimizer} is a solver written in C. It has simple,
well-documented library interfaces available in C, C++, Java, .NET and Python,
as well as an extremely detailed reference manual. Unlike SCIP, Gurobi is a
commercial product, but the company provides unlimited free academic licenses
which are unrestricted in terms of model size. The most recent release was
version 4.6 in November 2011. The earliest release I am able to find is Gurobi
2.0 in August 2009.

Judging by release dates, neither solver is significantly more established
than the other. Tests conducted at Arizona State University indicate that
Gurobi is consistently the highest performing solver of the seven tested
\cite{SolversPerformance_2012}.
