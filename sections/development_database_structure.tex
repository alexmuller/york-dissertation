%!TEX root = ../Project.tex

\subsection{Database structure}

[@todo what normal form is this in?]

As relational databases are so popular and are supported in many of the
software frameworks I look at in Section~\ref{sec:webframeworks}, there is no
advantage to choosing a database model apart from relational.

Web frameworks abstract away anything related to a specific database product
using an \gls{orm}, so the module allocation application can be written
without regard for the final database product that \gls{itservices} will use
to deploy the application.

Preliminary conversations with departments revealed additional requirements
for the system which informed the database structure. There are several
related entities stored by the University outside of this system, which have
the properties given in Table~\ref{development_database_uni_entities}.

\begin{table}
  \begin{tabular}{ | l | l | }
    \hline
    \textbf{Entity} & \textbf{Properties} \\
    \hline
    Student    & Username (unique), \gls{routecode}, \gls{stage} \\
    Module     & Module code (unique), module name, number of credits, maximum class size \\
    \hline
  \end{tabular}
  \caption{Entities and their properties as stored by the University of York}
  \label{development_database_uni_entities}
\end{table}

Each module is given a number of credits representing the estimated workload
for that module, and each student must take 120 credits every academic year.

The two pilot departments currently use paper forms, copies of which are given
in [@todo Appendix~ xx]. In order to create a scalable system that could be
used across any number of departments, the project author first generalised
the paper forms provided by the pilot departments.

This generalisation involved the following observations:

\begin{itemize}
  \item Each degree type (single subject, joint honours, etc) receives a different form
  \item Each form contains one or more groups, each of which contains one or more modules
\end{itemize}

\subsubsection{Entity-relationship diagram}

Figure~\ref{er_diagram} is an entity-relationship diagram of the module
allocation system.

\usetikzlibrary{positioning}
\tikzstyle{every relationship} = [draw=black, fill=black!10, text=black]
\begin{landscape}
\begin{figure}
  \begin{tikzpicture}[node distance=8em]
    
    \node[entity] (student) {Student};
      \node[attribute] (studentid) [above=1cm of student] {\key{Identifier}} edge (student);
      \node[attribute] (routecode) [above left=1cm of student] {Route code} edge (student);
    
    \node[relationship] (makes) [below left of=student] {Makes} edge (student);
    
    \node[entity] (choice) [below of=makes] {Choice} edge (makes);
      \node[attribute] (rank) [left=1cm of choice] {Rank} edge (choice);
    
    \node[relationship] (given) [below right of=student] {Given} edge (student);
    
    \node[entity] (allocation) [below of=given] {Allocation} edge (given);
    
    \node[relationship] (allocof) [below of=allocation] {Of} edge (allocation);
    \node[relationship] (choiceof) [below of=choice] {Of} edge (choice);
    
    \node[entity] (moduleav) [below left of=allocof] {Module Availability};
      \node[attribute] (studentmin) [below left=0cm and 1cm of moduleav] {Student min} edge (moduleav);
      \node[attribute] (studentcap) [below left=1.2cm and 1cm of moduleav] {Student cap} edge (moduleav);

    \draw[link] (choiceof) |- node [pos=0.05, auto, swap] {} (moduleav);
    \draw[link] (allocof) |- node [pos=0.05, auto, swap] {} (moduleav);

  \end{tikzpicture}
  \caption{(Incomplete) Entity-relationship diagram for the module allocation system.}
  \label{er_diagram}
\end{figure}
\end{landscape}

% Helpful demo/test document

% \begin{landscape}
% \tikzstyle{every entity} = [top color=white, bottom color=blue!30, draw=blue!50!black!100]
% \tikzstyle{every attribute} = [top color=white, bottom color=yellow!20, draw=yellow, node distance=1cm]
% \begin{tikzpicture}[node distance=1.5cm, every edge/.style={link}]
%   \node[entity] (emp) {Employee};
%   \node[isa] (isa) [below=1cm of emp] {ISA} edge (emp);
%   \node[entity] (mec) [below left=1cm of isa] {Mechanic} edge (isa);
%   \node[entity] (sal) [below right=1cm of isa] {Salesman} edge (isa);      %%%%%%%%%%
%   \node[relationship] (does) [left=of mec] {Does} edge (mec);
%   \node[weak entity] (rep) [below=of does] {RepairJob} edge (does);
%   \node[ident relationship] (reps) [below=of rep] {Repairs} edge [total] (rep);
%   \node[entity] (car) [right=of reps] {Car} edge [<-] (reps);
%   \node[relationship] (buy) [below=of sal] {Buys};                         %%%%%%%%%%
%   \node[relationship] (sel) [right=of buy] {Sells};                        %%%%%%%%%%
%   % \node[derived attribute] (sval) [right=of sel] {Value} edge (sel);
%   \draw[link] (car.10) -| (buy) (buy) edge (sal);
%   \draw[link] (car.-10) -| (sel) (sel) |- (sal);
%   \node[entity] (cli) [below right=0.5cm and 3.7cm of car] {Client};
%   % \node[multi attribute] (cphone) [below right=of cli] {Phone} edge (cli);
%   \draw[link] (cli.70) |- node [pos=0.05, auto, swap] {buyer} (sel);
%   \draw[link] (cli.110) |- node [pos=0.05, auto] {seller} (buy);
% \end{tikzpicture}
% \end{landscape}

