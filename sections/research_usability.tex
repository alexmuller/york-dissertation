%!TEX root = ../Project.tex

\subsection{Usability}

The rather broad term usability refers to how easy a system (or indeed
anything) is to use, though this report obviously only discusses the usability
of software in general and web applications in particular.

% http://www.faqs.org/docs/artu/ch11s01.html

One guideline that is useful to bear in mind when developing a piece of
software is the ``Rule of Least Surprise'' \cite{RaymondAUP_2003} (also
referred to as the ``Principle of Least Astonishment''). Raymond writes that
every piece of software should at every time simply ``do the least surprising
thing'', as the lack of surprises allows the user to spend less time worrying
about the application and instead allows them to focus on what they are trying
to use the software for. His advice is that no application should force the
user to create an entirely new mental model to use the software, and it should
instead attempt to mimic existing interfaces or applications that the user
might be familiar with.

Raymond's advice is clear and useful for anybody who has observed a
non-technical user attempt to use a piece of software with poor usability. I
can recall friends and relatives struggling with dialog boxes and error
messages that they were not expecting, which often appear at precisely the
wrong time.

Another factor of usability is the process around saving state in the
application. In the past, web applications would \gls{post} data to the web
server only when the user explicitly clicked a submit button (written on the
web as \texttt{<input type=submit>}). Modern web applications, such as
Google's Gmail service, make use of JavaScript and low-latency Internet
connections to automatically save data to the server while the user is
working. Sandlund \cite{sandlund2009websoftware} explains that by
automatically saving data at regular intervals, there is far less chance of
data loss.

During Sandlund's user testing, participants responded positively to the
application automatically saving data. However, he decided not to include a
submit button in his application, and he notes that some users found the lack
of an explicit save button ``a bit disturbing''. It is important to include an
explicit ``Save'' button on an application that auto-saves -- one can imagine
the confusion and uncertainty caused if the user is unaware that their data is
being saved in the background.
