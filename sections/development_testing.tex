%!TEX root = ../Project.tex

\subsection{Testing}

% Unit tests?
% Performance testing?

% User testing

Prior to the application being made available to students, members of staff
and students were requested to test the application. A summary of the feedback
is given in Appendix~\ref{sec:testingfeedback}.

As a result of this feedback, several changes were made in advance of the
application being made accessible to students:

\begin{itemize}
  \item The form validation was improved so that elective credit values were properly checked
  \item Modules displayed on the confirmation screen were put in ascending order of rank
  \item The explanatory text was clarified with the help of departments and the \gls{aso}
\end{itemize}

% Algorithm testing

The departments provided their finalised courses, modules and students (making
use of randomly generated ``fake'' usernames) to the project implementer. He
was then able to perform an allocation based on this data with random rankings
created by the system, to ensure that the allocation algorithm performed as
expected.

The random rankings were generated during testing using the following
pseudocode:

\begin{verbatim}
  For each student in the sytem:
    Get the student's choices and clear them
    Get the student's groups
    For each of the student's groups:
      Get the list of modules in that group
      Randomise the order of the list
      Set rank = 1
      For each module in the randomly-ordered list:
        Assign it a value of 'rank'
        rank += 1
        Create a choice entity for this (student, module, rank)
    Save the student's choices
\end{verbatim}

Clearly students do not rank modules with absolute uniform randomness in
practice. Some modules are more popular than others, and as such will receive
more first choices. This will influence the objective function, but these
random rankings were used to provide useful information on whether the hard
constraints of the model could be met.

Initially, the data provided by the History department caused the solver to
report that finding any allocation, no matter what the value of the objective
function, was an insolvable problem. In this case, Gurobi reported that the
model was infeasible.

The project implementer noted that the \gls{iis} computed by Gurobi revealed
that modules with database ID from 1 to 8 were the problem modules -- the
solver was unable to find a solution such that the maximum class size
constraint could be met for these modules. These eight modules had these
properties in the database:

\begin{verbatim}
 id   code         min   max   credits
---------------------------------------
 1    HIS00002I    56    64    20
 2    HIS00006I    56    64    20
 3    HIS00013I    56    64    20
 4    HIS00015I    56    64    20
 5    HIS00058I    56    64    20
 6    HIS00059I    56    64    20
 7    HIS00064I    56    64    20
 8    HIS00065I    56    64    20
---------------------------------------
\end{verbatim}

On further inspection, it was revealed these eight modules form the
department's ``Histories and Contexts'' group. They are self-contained
modules; they only appear with all eight together and never with any other
modules. This means that it is possible to count exactly the number of
allocations that will be required to be made by the system.

The following table shows the seven groups that these modules appear in. For
each of these seven groups, the departments have specified a number of credits
that the system is required to allocate. As the modules are all worth 20
credits either one or two allocations needs to be made, depending on the
group.

\begin{verbatim}
 group_id  sheet_id  credits_to_alloc  mods_to_alloc  num_students  num_allocations
------------------------------------------------------------------------------------
 1         1         40                40 / 20 = 2             220    220 x 2 = 440
 3         2         20                20 / 20 = 1               1      1 x 1 =   1
 5         3         20                20 / 20 = 1               6      6 x 1 =   6
 7         4         20                20 / 20 = 1              11     11 x 1 =  11
 9         5         20                20 / 20 = 1              26     26 x 1 =  26
 11        6         20                20 / 20 = 1               7      7 x 1 =   7
 13        7         40                40 / 20 = 2              16     16 x 2 =  32
------------------------------------------------------------------------------------
 Total:                                                        287              523
\end{verbatim}

As each group has a fixed number of students associated with it, it is trivial
to see that 523 allocations \textbf{must} be made by the system in order to
meet the hard constraint that every student takes the correct number of
credits. However, each of the eight modules was given a cap of just 64
students, meaning the solver could only ever generate 512 allocations.

The History department was consulted and it was revealed they had been overly
cautious when setting the cap to allow sufficient places for visiting
students. In fact, the cap had been set so low that there was not enough space
for each student in the department to take their full quota of modules.
Increasing the cap on these eight modules from 64 to 66 meant that 528
allocations could be made. The allocation was tested with these new caps and,
as expected, it completed without problems.

The anonymised data provided by the Archaeology department did not cause the
solver to report any problems. As discussed in
Section~\ref{sec:algo_humanaspects}, Archaeology has far more available
capacity in their modules than History so this is unsurprising.

\begin{table}
  \begin{center}
    \begin{tabular}{ | l | l | l | l | l | l | }
      \hline
      \textbf{Sum of ranks} & \textbf{No cap} & \textbf{15/15 cap} & \textbf{12/6 cap} & \textbf{10/10 cap} & \textbf{8/4 cap} \\
      \hline
      1 & 81 & 80.5 & 83 & 83 & 78 \\
      2 & 225 & 223.5 & 220.5 & 231.5 & 214 \\
      3 & 39.5 & 45.5 & 45 & 31 & 47 \\
      4 & 20.5 & 17.5 & 20.5 & 21 & 31 \\
      5 & 4.5 & 3.5 & 2 & 3.5 & 1 \\
      6 & 132.5 & 155 & 141.5 & 140 & 127 \\
      7 & 58 & 44 & 54.5 & 49 & 57 \\
      8 & 11.5 & 7.5 & 15 & 15.5 & 36 \\
      9 & 6 & 2 & 5.5 & 8.5 & 0 \\
      10 & 3.5 & 5 & 2 & 7.5 & 0 \\
      11 & 3 & 1.5 & 0.5 & 0 & 0 \\
      12 & 2 & 2.5 & 1 & 0 & 0 \\
      13 & 1 & 1 & 0 & 0 & 0 \\
      14 & 0 & 0.5 & 0 & 0 & 0 \\
      15 & 0.5 & 1.5 & 0 & 0 & 0 \\
      16 & 0 & 0 & 0 & 0 & 0 \\
      17 & 0 & 0 & 0 & 0 & 0 \\
      18 & 1 & 0 & 0 & 0 & 0 \\
      19 & 0 & 0 & 0 & 0 & 0 \\
      20 & 0 & 0 & 0 & 0 & 0 \\
      21 & 0 & 0 & 0 & 0 & 0 \\
      22 & 0 & 0 & 0 & 0 & 0 \\
      23 & 0 & 0 & 0 & 0 & 0 \\
      24 & 0.5 & 0 & 0 & 0 & 0 \\
      25 & 1 & 0 & 0 & 0 & 0 \\
      \hline
    \end{tabular}
  \end{center}
  \caption{Data obtained from allocations made to test the equity cap}
  \label{gurobi_equity_table}
\end{table}

The project implementer also tested the additional hard constraint, called
``equity across students'', to see whether it provided a better allocation.

Table~\ref{gurobi_equity_table} shows data obtained from some test allocations
made to demonstrate the effect that this hard constraint has. Each column in
the table represents a different equity cap. For each equity cap, random
rankings were generated and an allocation was performed on those rankings.
This was performed twice for each cap and the figures were averaged to obtain
the result shown. The cap is given in terms of two figures; the first figure
is the cap for single-subject students, and the second is the cap for joint
students. This is due to the fact joint students take less modules, and as
such would expect a lower cap.

With no cap present, we observe one or two students who receive incredibly bad
allocations. As the cap is reduced, clearly the number of bad allocations is
reduced. However, as might be expected we also start to observe the number of
perfect allocations decreasing.
