%!TEX root = ../Project.tex

\subsection{Testing}

\subsubsection{User acceptance testing}

Several days prior to the application being made available to students,
members of staff and students were requested to test the application (similar
to the small-scale ``friends and family'' that Kuniavsky
\cite{kuniavsky2003userexperience} puts forward). A summary of the points
mentioned is given in Appendix~\ref{sec:testingfeedback}. As a result of this
feedback, several changes were made in advance of the application being made
accessible to students:

\begin{itemize}
  \item The form validation was improved so that constraints on elective
        credit values were properly enforced (for example, so that no student can
        take a negative number of elective credits)
  \item Modules displayed on the confirmation screen
        (Figure~\ref{walkthrough_student_confirm}) were put in ascending order of
        rank to make it easier for students to spot any inconsistencies
  \item The explanatory text for students was clarified and expanded in places
        with the help of the pilot departments and \gls{lc} in the \gls{aso}
\end{itemize}

\subsubsection{Allocation testing}

The departments provided their finalised courses, modules and students (making
use of randomly generated ``fake'' usernames to comply with data protection
legislation) to the project implementer. I was then able to perform an
allocation based on this data with random rankings created by the system, to
ensure that the allocation problem given to the solver returned expected
results. The random rankings were generated during testing using the following
pseudocode:\mynobreakpar

\begin{Verbatim}[samepage=true]
  For each student in the system:
    Get the student's choices and clear them
    Get the student's groups
    For each of the student's groups:
      Get the list of modules in that group
      Randomise the order of the list
      Set rank = 1
      For each module in the randomly-ordered list:
        Assign it a value of 'rank'
        rank += 1
        Create a choice entity for this (student, module, rank)
    Save the student's choices
\end{Verbatim}

Clearly students do not rank modules with absolute uniform randomness in the
real world -- some modules are more popular than others based on content and
lecturer, and as such will receive more first choices. This will influence the
objective function, but these random rankings were used to test whether the
hard constraints of the model could be met.

Initially, the course and module data provided by the History department
caused the solver to report that finding any allocation, no matter what the
value of the objective function, was an insolvable problem. In this case,
Gurobi simply reported that the model was infeasible and it was impossible to
allocate all students the requisite number of modules. The project implementer
noted that the \gls{iis} computed by Gurobi (a subset of the variables and
constraints that helps find a solvable model) revealed that modules with
database ID from 1 to 8 were the problem modules -- the solver was unable to
find a solution such that the maximum class size constraint could be met for
these modules. These eight modules had similar properties in the
database:\mynobreakpar

\begin{Verbatim}[samepage=true]
 id   code         min   max   credits
---------------------------------------
 1    HIS00002I    56    64    20
 2    HIS00006I    56    64    20
 3    HIS00013I    56    64    20
 4    HIS00015I    56    64    20
 5    HIS00058I    56    64    20
 6    HIS00059I    56    64    20
 7    HIS00064I    56    64    20
 8    HIS00065I    56    64    20
---------------------------------------
\end{Verbatim}

On further inspection, it was revealed these eight modules form the
department's ``Histories and Contexts'' group. They are entirely
self-contained modules; they only appear with all eight of them together and
never with any other modules. This means that it is possible to count exactly
the number of allocations that will be required to be made by the system. The
following data shows the seven groups that these modules appear in. For each
of these seven groups, the departments have specified a number of credits that
the system is required to allocate. As the modules are all worth 20 credits
either one or two allocations always needs to be made, depending on the group.

\begin{Verbatim}[samepage=true]
 group_id  sheet_id  credits_to_alloc  mods_to_alloc  num_students  num_allocations
------------------------------------------------------------------------------------
 1         1         40                40 / 20 = 2             220    220 x 2 = 440
 3         2         20                20 / 20 = 1               1      1 x 1 =   1
 5         3         20                20 / 20 = 1               6      6 x 1 =   6
 7         4         20                20 / 20 = 1              11     11 x 1 =  11
 9         5         20                20 / 20 = 1              26     26 x 1 =  26
 11        6         20                20 / 20 = 1               7      7 x 1 =   7
 13        7         40                40 / 20 = 2              16     16 x 2 =  32
------------------------------------------------------------------------------------
 Total:                                                        287              523
\end{Verbatim}

As each group has a constant number of students associated with it, it is
trivial to see that 523 allocations \textbf{must} be made by the system in
order to meet the hard constraint that every student takes the correct number
of credits. However, each of the eight modules was given a cap of just 64
students, meaning the solver could only ever generate 512 allocations for
these modules. The History department was consulted and it was revealed they
had been overly cautious when setting the cap, to allow sufficient places for
visiting students. In fact, the cap had been set so low that there was not
enough space for each student in the department to take their full quota of
modules. Increasing the cap on these eight modules from 64 to 66 meant that
528 allocations could be made. The allocation was tested with these new caps
and, as expected, it completed without problems.

The anonymised data provided by the Archaeology department did not cause the
solver to report any problems with regards to hard constraints. As discussed
in Section~\ref{sec:algo_humanaspects}, Archaeology has far more available
capacity in their modules than History so this result is unsurprising.

\begin{table}
  \begin{center}
    \begin{tabular}{ | l | l | l | l | l | l | }
      \hline
      \textbf{Sum of ranks} & \textbf{No cap} & \textbf{15/15 cap} & \textbf{12/6 cap} & \textbf{10/10 cap} & \textbf{8/4 cap} \\
      \hline
      1 & 81 & 80.5 & 83 & 83 & 78 \\
      2 & 225 & 223.5 & 220.5 & 231.5 & 214 \\
      3 & 39.5 & 45.5 & 45 & 31 & 47 \\
      4 & 20.5 & 17.5 & 20.5 & 21 & 31 \\
      5 & 4.5 & 3.5 & 2 & 3.5 & 1 \\
      6 & 132.5 & 155 & 141.5 & 140 & 127 \\
      7 & 58 & 44 & 54.5 & 49 & 57 \\
      8 & 11.5 & 7.5 & 15 & 15.5 & 36 \\
      9 & 6 & 2 & 5.5 & 8.5 & 0 \\
      10 & 3.5 & 5 & 2 & 7.5 & 0 \\
      11 & 3 & 1.5 & 0.5 & 0 & 0 \\
      12 & 2 & 2.5 & 1 & 0 & 0 \\
      13 & 1 & 1 & 0 & 0 & 0 \\
      14 & 0 & 0.5 & 0 & 0 & 0 \\
      15 & 0.5 & 1.5 & 0 & 0 & 0 \\
      16 & 0 & 0 & 0 & 0 & 0 \\
      17 & 0 & 0 & 0 & 0 & 0 \\
      18 & 1 & 0 & 0 & 0 & 0 \\
      19 & 0 & 0 & 0 & 0 & 0 \\
      20 & 0 & 0 & 0 & 0 & 0 \\
      21 & 0 & 0 & 0 & 0 & 0 \\
      22 & 0 & 0 & 0 & 0 & 0 \\
      23 & 0 & 0 & 0 & 0 & 0 \\
      24 & 0.5 & 0 & 0 & 0 & 0 \\
      25 & 1 & 0 & 0 & 0 & 0 \\
      \hline
    \end{tabular}
  \end{center}
  \caption{Data obtained from allocations made to test the equity cap}
  \label{gurobi_equity_table}
\end{table}

The additional hard constraint, called ``equity across students'', was also
tested to see whether it provided a better allocation.
Table~\ref{gurobi_equity_table} shows data obtained from some test allocations
made to demonstrate the effect that this hard constraint has. Each column in
the table represents a different equity cap. For each equity cap, random
rankings were generated and an allocation was performed on those rankings.
This was performed twice for each cap and the figures were averaged to obtain
the result shown. The cap is given in terms of two figures; the first figure
is the cap for single-subject students, and the second is the cap for joint
students. This is due to the fact joint students take fewer modules, and as
such would expect a lower cap. With no cap present, we observe one or two
students who receive incredibly bad allocations (a sum of 25 across four
modules indicates a student receiving their 5th and 7th choice in one group
and their 6th and 7th in a second group). As the cap is reduced, clearly the
number of bad allocations is reduced. However, as might be expected we also
start to observe the number of perfect allocations decreasing.
