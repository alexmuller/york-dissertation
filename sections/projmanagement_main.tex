%!TEX root = ../Project.tex

\section{Project management and requirements}
\label{sec:requirements}

The \gls{aso} identified pilot departments who would most benefit from using
the software to allocate optional modules. A project steering group was formed
consisting of representatives from each of the departments (Archaeology and
History), administrative staff responsible for IT and timetabling, and the
project author and supervisor. The membership of the project steering group is
as follows:

\begin{itemize}
  \item Project implementer (Alex Muller)
  \item Project supervisor (James Cussens, Department of Computer Science)
  \item Project manager (\gls{lc}, \gls{aso})
  \item Head of Enterprise Systems (\gls{itservices})
  \item University Timetabling Officer (Campus Services)
  \item Assistant Manager, Student Systems (\gls{mh}, Registry Services)
  \item Chair of Board of Studies (\gls{sa}, Department of Archaeology)
  \item Departmental administrator (\gls{cm}, Department of Archaeology)
  \item Student representative (Department of Archaeology)
  \item Chair of Board of Studies (Department of History)
  \item Departmental administrator (Department of History)
  \item Student representative (Department of History)
\end{itemize}

Basic requirements for the project were given in the initial project
specification. At the first steering group meeting in May 2011 and later that
year this group reinforced and prioritised certain requirements. When the
project commenced, the requirements could be summarised as:

\begin{quote}
  To create a system to collect choices and allocate optional modules to
  university students. The system should be user-friendly and accessible via a
  web browser for both students and staff. It should provide benefits over the
  current paper-based method of allocating modules, such as:
  
  \begin{itemize}
    \item Reduced administrative overhead (easier for departments)
    \item Improved student experience (more convenient)
    \item Increased level of fairness in allocation (as judged by students and departments)
    \item Transparency to students in the way modules are allocated
    \item Providing summary data on choices and allocations to departments
  \end{itemize}
  
  At a minimum, the system should handle ranking of choices. It should not be
  first-come, first-served, as this results in students having to use the
  system at a specific time.
  
  The system should be flexible in terms of:
  
  \begin{itemize}
    \item Number of modules
    \item Number of students
    \item Number of constraints
  \end{itemize}
  
  The system must, once this project is complete, be owned and managed by
  somebody at the University of York, most probably somebody working in
  \gls{itservices}.
  
\end{quote}

At the initial meeting, two items were given as potential features, but not
requirements:

\begin{itemize}
  \item To more accurately gauge student preference (for instance weighting rather than ranking)
  \item To keep a memory, to compensate students if they had a bad allocation the previous year
\end{itemize}

The project is also monitored by the \gls{sipig}, a group consisting of
academic and administrative staff at the University of York who are
responsible for overseeing the progress of projects related to central
services. \gls{sipig} is chaired by the ``Director of Information and
University Librarian'' and projects are usually carried out by the University
Library \& Archives and \gls{itservices}.
