%!TEX root = ../Project.tex

\section{Conclusions}
\label{sec:conclusions}

This section contains conclusions drawn at the end of the project, including
reviewing the research undertaken and whether it was sufficient, the
application itself, and summarises the pilot departments' use of the
application, including their feedback.

% This is similar to the abstract. The difference is that you should assume
% here that the reader of the conclusions has read the rest of the report.

% Talk about the methodology you adopted and why you chose. Make sure you
% mention what worked and what did not - explain why you think things did not
% work or could be improved - this shows learning from the experience. This
% section should be a summary of everything you have done with your personal
% comments and findings and perhaps some discussion on whether you agree with
% the references you cited.

\subsection{Research}

While the research carried out formed a strong basis for creating a web
application, the lack of time dedicated to implementation meant that some
parts of the research could not be put into practice as much as I would have
liked. Additionally it would have been beneficial to spend far more time with
the project clients and end users in order to gather more detailed
requirements and perfect the application interface.

Furthermore, it became clear during implementation that there were areas I
should have researched more thoroughly, particularly database theory. This was
not helped by my unfamiliarity with the framework and associated tools -- in
the case of the Spring framework, database tables were automatically created
from the entities defined in the application logic. Spending more time testing
and playing with this before implementation started would have increased my
understanding and could have produced a higher quality application in less
time.

Early in the research period (before implementation started) I decided to
demonstrate progress to clients by creating simple \gls{html} prototypes that
had no logic behind them. Showing these prototypes to the group may have
resulted in the non-technical stakeholders believing that implementation was
further progressed than it actually was. I would still create prototypes in
the future as I believe they are the best way to involve a client in the
development process, but I would more carefully explain that it is simply a
prototype with no application code working behind it. Fundamentally, I believe
that providing mockups and prototypes to clients throughout the research
period was hugely important in allowing them to engage with the design of the
application, and as a result produced better quality software.

\subsection{Application}

The implementation of the application should have begun earlier than it did.
The increased amount of time would have provided a higher quality product and
would have reduced the stress caused by such a time-constrained implementation
period. Requirements gathering could have been completed earlier with the help
of members of the staff in the departments and earlier involvement from
\gls{itservices} may even have reduced the total amount of time that they
spent providing input for the project.

Although there was less time available for implementation than I would have
liked, I feel I have met one the key requirements that is somewhat hard to
measure: maintainability. Code quality in places is not as good as one might
expect from a bespoke (purchased) product, but as the software is written
using a framework that University developers are familiar with, I would hope
that they would be able to understand the structure of the application and
improve the code in far less time than it would take to create a similar
application from scratch -- \gls{itservices} developers have far more
experience with creating Java web applications than I do, but the constrained
optimisation part of this project is, I believe, a hugely important piece that
might they might be unfamiliar with. The code created for this application
will give any future Java developers an understanding of Gurobi, and I look
forward to walking through this part of the code with them if they would like
to. The \gls{itservices} developer assisting with the application commented
that he could think of several other areas of University software where
knowledge of constrained optimisation and linear programming might be helpful.

As mentioned in Section~\ref{sec:issuesarising}, if undertaking a project
similar to this in the future I might first create a high-fidelity prototype
in a language I was very familiar with in order to meet the basic project
requirements. I would then look at re-implementing the application in a
language that would allow it to be maintained by other developers as required.
It would be possible to reuse certain aspects of the prototype (for example,
all of the requirements gathering and some of the front-end code), and I am of
the opinion that this second implementation would be of higher quality, as
mistakes made in the prototype would not be repeated. Whether or not this
approach is suitable would depend on the nature of the project; as the module
allocation software was fairly isolated from other systems and was a
relatively small application in all, I think this would have been a good
approach to take. There was a trade-off made between my inexperience with the
chosen software framework (and therefore the amount of time implementation
took) and the usefulness of the application to \gls{itservices}. In this case,
I believe the trade-off was beneficial to the project - I managed to create a
working---though not perfect---application, and the code will still be of some
use to the University in the future. However, one can imagine a project like
this resulting in no useful software being created. This is the reason for my
recommendation to implement twice, once using an already familiar language,
when possible.

\subsection{Pilot and feedback from departments}

These underlying issues notwithstanding, the trial of the software was a
definite success. During the week when students selected their modules very
few issues arose, and those that did were mitigated with the help of
\gls{itservices} and the departmental administrators. The application was
received positively by those students who gave feedback to the departments on
the module selection process. There were problems caused by the security
requirement that the application not be available outside the University
campus. Students who attempted to use the \gls{vpn} to access the application
were unable to, and this resulted in departmental administrators having to
input those students' choices after receiving them by email (7 out of 204 in
the case of Archaeology). This issue would require further investigation by
those with access to relevant log files to resolve.

An area I would like to have spent more time improving is the allocation
process and the associated constraints. If running this project again, I would
be more firm with eliciting allocation-related requirements from departments
earlier in the process, helping them to be decisive in quantifying soft
constraints (such as the coefficients used in the objective function). This
should be easier in future years as this pilot has provided some results which
can be iterated on with feedback from the departments.

This trial has also provided departments with data that they had never had
access to in the past -- the History department, with almost one thousand
undergraduates in total, had never before been able to analyse the choices
that students made in order to make decisions about the courses they offer.
The department felt that the data provided by this application would be
incredibly useful in allowing them to quanity the popularity of their modules
and would inform decisions about staffing and resourcing made in the future.

Unfortunately, some of the time savings that this application would have
provided were offset by time that departments had to spend during requirements
gathering and training sessions, learning to use the new software. These time
savings will be seen in future years if the software is used again. The
biggest time saving was in the software creating a \gls{csv} file of
allocations for \gls{ssdt} to process. In previous years the departmental
administrators would have to input each student's modules one by one. The
Archaeology department estimated they had saved one day's work with this file
being generated by the software, and the History department several days
because of their larger student population.

The Archaeology department commented that out of the 204 students who used
this system, just three contacted \gls{cm} to indicate they were unhappy with
the modules they had been allocated. \gls{cm} commented that she had a feeling
this was ``fewer than usual, but [that she] couldn’t back that up with
anything''. Due to the deadline for this report I am unable to provide any
student feedback on the allocation from the History department, though the
staff felt that the allocation, with some manual modifications, was as good as
they could have managed by hand.
