%!TEX root = ../Project.tex

\section{Conclusions}
\label{sec:conclusions}

This section contains conclusions drawn at the end of the project, including
reviewing the research undertaken and whether it was sufficient, the
application itself, and how the pilot departments' use of the application
went.

% This is similar to the abstract. The difference is that you should assume
% here that the reader of the conclusions has read the rest of the report.

% Talk about the methodology you adopted and why you chose. Make sure you
% mention what worked and what did not - explain why you think things did not
% work or could be improved - this shows learning from the experience. This
% section should be a summary of everything you have done with your personal
% comments and findings and perhaps some discussion on whether you agree with
% the references you cited.

\subsection{Research}

While the research carried out formed a strong basis for creating a web
application, the lack of time dedicated to implementation meant that some
parts of the research could not be put into practice as much as I would have
liked. For example, it would have been beneficial to spend far more time with
the project clients and end users in order to perfect the interface.

Furthermore, it became clear during implementation that there were areas I
should have researched more thoroughly, particularly database theory. This was
not helped by my unfamiliarity with the framework and associated tools -- in
the case of the Spring framework, database tables were automatically created
from the entities defined in the application logic. Spending more time testing
and playing with this before implementation started would have increased my
understanding and might have produced a higher quality application in less
time.

Early in the research period (before implementation started) I decided to
demonstrate research progress to clients by creating simple \gls{html}
prototypes that had no logic behind them. Showing these prototypes to the
group may have resulted in the non-technical stakeholders believing that
implementation was further progressed than it actually was. I would still
create prototypes in the future as I believe they are the best way to involve
a client in the development process, but I would more carefully explain that
it is simply a prototype with no application code working behind it.
Fundamentally, I believe that providing mockups and prototypes to clients
throughout the research period was hugely important in allowing them to engage
with the design of the application, and as a result produced better quality
software.

\subsection{Application}

Although there was less time available for implementation than I would have
liked, I feel I have met one the key requirements that is hard to quantify:
maintainability. The software would require quite some work if it was to be
maintained centrally by the University. Code quality in places is not as good
as one might expect from a bespoke (purchased) product, but as the software is
written using a framework that University developers are familiar with, I
would hope that they would be able to understand the structure of the
application and improve the code in far less time than it would take to create
a similar application from scratch -- \gls{itservices} developers have far
more experience with creating Java web applications than I do, but the
constrained optimisation part of this project is, I believe, a hugely
important piece that might they might be unfamiliar with. The code created for
this application will give any future Java developers an understanding of
Gurobi.

As mentioned in Section~\ref{sec:issuesarising}, if undertaking a project
similar to this in the future I would first create a high-fidelity prototype
in a language I was very familiar with in order to meet the basic project
requirements. I would then look at re-implementing the application in a
language that would allow it to be maintained by other developers as required.
It would be possible to re-use certain aspects of the prototype (for example,
some of the front-end code), and I am of the opinion that this second
implementation would be of higher quality, as mistakes made in the prototype
would not be repeated. Whether or not this approach is suitable would depend
on the nature of the project; as the module allocation software was fairly
isolated from other systems and was a relatively small application in all, I
think this would have been a good approach to take.

There was a trade-off made between my inexperience with the chosen software
framework (and therefore the amount of time implementation took) and the
usefulness of the application to \gls{itservices}. In this case, I believe the
trade-off was beneficial to the project - I managed to create a
working---though not perfect---application, and the code will still be of some
use to the University. However, one can imagine a project like this resulting
in no useful software being created. This is the reason for my recommendation
to implement twice, once using a familiar language, when possible.

\subsection{Pilot}

These underlying issues notwithstanding, the trial of the software was
definitely a success. During the week when students selected their modules
very few issues arose, and those that did were mitigated with the help of the
departmental administrators. The application was received positively by
students, who gave feedback to the departments on the module selection
process.

\emph{[@todo: Conclusions on how the allocation went]}

% Time savings offput by helping with application

An area I would like to have spent more time improving is the allocation and
associated constraints. If running this project again, I would be more firm
with eliciting allocation-related requirements from departments earlier in the
process, helping them to be decisive in quantifying soft constraints (such as
the coefficients used in the objective function). This should be easier in
future years as this pilot has provided some results which can be improved
with feedback from the departments.

Due to the deadline for this report I am unable to comment on whether students
were satisfied with the modules they were allocated, but the departments
appeared happy with the allocation that was produced.
