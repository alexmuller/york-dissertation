%
%  Allocating optional modules to University of York
%  students using constrained optimisation
%
%  BSc Computer Science/Maths final-year dissertation
%
%  Created by Alex Muller on 2011-10-10.
%  Copyright (c) 2011 Alex Muller. All rights reserved.
%
\documentclass[]{scrartcl}
% article, report, scrartcl, uoy_cs/UoYCSproject ?

\usepackage[utf8]{inputenc} % Use utf-8 encoding for foreign characters

% Pages and margins
\usepackage{fullpage}
\usepackage[top=2cm, bottom=4cm, left=2cm, right=2cm]{geometry}

\setlength{\parskip}{1ex plus 0.5ex minus 0.2ex}
% \addtolength{\parskip}{\baselineskip}
% \usepackage{parskip}

% Running Headers and footers
%\usepackage{fancyhdr}

% Multipart figures
%\usepackage{subfigure}

% More symbols
%\usepackage{amsmath,amssymb,latexsym}

\usepackage{boxedminipage} % Surround parts of graphics with box
\usepackage{listings} % Package for including code in the document
\usepackage{lastpage} % Number of pages

% This is now the recommended way for checking for PDFLaTeX:
\usepackage{ifpdf}

\ifpdf
\usepackage[pdftex]{graphicx}
\else
\usepackage{graphicx}
\fi

\ifpdf
\usepackage[pdftex]{hyperref}
\else
\usepackage{url}
\fi

% Entity-relationship diagrams
\usepackage{libs/tikz-er2}

% Glossary
\usepackage{glossaries}
\makeglossaries
% Entries

% \newacronym[\glsshortpluralkey=cas,\glslongpluralkey=contrived
% acronyms]{aca}{aca}{a contrived acronym}

\newglossaryentry{itservices}{
  name = {IT Services},
  description = {is a
    support service that, ``together with the Library and Archives, forms the
    Information Directorate" at the University of York. It is responsible for,
    among other things, computer hardware and software, infrastructure and web
    services at the University}
}

% Acronyms

\newacronym{oscon}{OSCON}{The O'Reilly Open Source Convention}
\newacronym{dbms}{DBMS}{Database Management System}
\newacronym{html}{HTML}{HyperText Markup Language}
\newacronym{sso}{SSO}{single sign-on}

% Title page stuff

\title{Allocating optional modules to University of York students using constrained optimisation}
\author{Alexander Muller}
\date{\today}

\begin{document}

\ifpdf
\DeclareGraphicsExtensions{.pdf, .jpg, .tif}
\else
\DeclareGraphicsExtensions{.eps, .jpg}
\fi

\maketitle

This is the report for a Bachelor of Science final-year project in Computer
Science and Mathematics at the University of York. The project was supervised
by Dr James Cussens, Senior Lecturer in the Artifical Intelligence Group,
Department of Computer Science.

This report is 3,000 words, as counted by running \texttt{detex <report.tex> |
wc -w}. It is \pageref{LastPage} pages long.

% The limits are 35,000 words and 70 pages - neither limit may be exceeded.
% Other projects I've seen have been 12,000/60, 11,000/60, etc

\newpage

\begin{abstract}
  % Just a couple of paragraphs summarising the report content. Must be
  % comprehensible to someone who has not read the rest.
  % Not more than 200 words - this is 160.
  From their second year onwards, most students at the University of York can
  choose between two or more optional modules to tailor their academic career,
  in the hope it will be more relevant, interesting and useful to them.
  Optional module allocation in most departments is currently handled using a
  paper form which must be returned to departmental administrators. This project
  aims to design and implement a piece of web-based software that can be used by
  departments and students to allocate modules more fairly and with less
  administrative overhead.
  The web application will be piloted by the Archaeology and History departments
  in March 2012 and, if successful, will be offered to all departments and
  maintained centrally by the University.
  This report discusses the choices made around the technology used, the
  development methodology and details relating to the allocation algorithm.
  The software will be evaluated according to criteria set out by the project
  steering group, and the resulting evaluation is also discussed.
\end{abstract}

\newpage

\tableofcontents

\newpage

\section{Statement of ethics}

% Informed consent

Those people volunteering to help with the project (interviewed during the
research stage) will at no point be put in a position of physical danger.
Consent will be obtained from all volunteers prior to their interview, and
volunteers' personal information will not be published or shared. A copy of
the consent form signed by all volunteers is given in
Appendix~\ref{sec:consent}.

% Do no harm

As far as the project author is aware, there are no immediate ethical issues
relating to the creation of the module allocation software.

% Confidentiality of data

The project steering group has noted that as a student, the author must not be
given access to any sensitive personal information. This includes, but is not
limited to, student names, email addresses and degree course information.
Development and testing of the software will be carried out with data that it
is in a similar form to real data held by the University. The University's
Data Protection Officer was consulted during the project, and their input is
discussed in Section~\ref{sec:dataprotection}.

\section{Introduction}

% The scope of the project, setting the scene for the remainder of the report.

This project is the result of requests by departments at the University of
York for more flexibility in the way they offer optional modules to
undergraduate students. The project is sponsored by University Teaching
Committee and is overseen in that regard by Laura Crossley of the Academic
Support Office.

As well as being marked as an undergraduate project, the software will be
evaluated independently by Ms Crossley. If it is judged to provide sufficient
advantages over the current methods of module allocation, it may be
recommended that responsibility for the software is handed to \gls{itservices}.

A project steering group was formed consisting of representatives from each of
the pilot departments (Archaeology and History), administrative staff
responsible for IT and timetabling, and the project author and supervisor. At
the initial meetings, this group set out the scope for the project…

\subsection{The current state of module allocation}

% Refer to documents provided by Laura
% (1) At the University of York

Universities worldwide want to give their students the most interesting and
useful education possible, and one way this can be accomplished is by offering
flexibility in modules students can take.

At the University of York, module allocation is something dealt with at the
departmental level. Some departments make use of centrally offered software
(eVision and SITS, the university management information system), though
several departments feel the software is not flexible enough and continue to
use paper-based forms that must be filled out and returned to the departmental
office. A smaller number of departments, such as Computer Science, have
written their own module choice and allocation software.

At some universities in the United Kingdom (for example, Warwick and Leeds),
enrolement is completed online, on a first-come, first-served basis.

% (2) Elsewhere around the world

\subsection{Web applications at the University of York}

% Student portal
% Timetabling gateway
% Google Apps for Education

The University is constantly improving the quality of web software available
to students and staff.

For the last few years, a single sign-on (SSO) solution has been provided
centrally, using Shibboleth. This improves both user experience and security,
as users are encouraged to only ever enter their University username and
password into one screen, where the web address will always begin
\texttt{https://shib.york.ac.uk/}.

In 2011, a new ``student portal'' was released, allowing students to view
personalised information relevant to them in one place. The application was
created by IT Services and was written in Java.

For the beginning of the 2011--12 academic year, new timetabling software is in
use to give members of the University access to their complete timetable in
one location---something which has not been possible before.

In June 2012, the University will move all students to Google's Apps for
Education product for email and calendaring, one of the aims of which is to
improve the user experience dramatically over the current webmail software.

\section{Research}

% One or more review chapters, describing the research you did at the
% beginning of the project period.

As web application development is an area that requires solving problems in
many different areas of Computer Science, the research completed in this phase
of the project was wide-ranging.

\subsection{Development methodology}

% Agile?

Zhang and Chung \cite{MODFM_2003} note that prototyping can be used to
reinforce client confidence as well as making better use of the time allocated
to development and implementation.

Bochicchio and Paiano \cite{PrototypingWebApplications_2000} note that
creating prototypes of web applications has several advantages.
The most relevant advantage for this project is that a mockup can be used to
get feedback from non-technical stakeholders such as the project manager and
the departmental contacts.

Prototypes can be defined as being low or high fidelity depending on how much
they are designed to resemble the final application. Low fidelity prototypes
can be creating using a marker pen and sheets of blank paper, whereas higher
fidelity prototypes may be coded to appear in a web browser and allow the user
to interact as they might with the final system.

\subsection{Database design}

A \gls{dbms} is a piece of software that manages the
database, including providing the ability to add or edit records stored in the
database. The \gls{dbms} will be one of the more mature products used in the
creation of this system, with many having been available since the 1990s.

Relational databases are a common feature of web applications. We note
that the University of York already deploys MySQL and Oracle (both relational
model) database systems for its web applications and MIS.

When designing a database, Johnson \cite{DatabaseModelsLanguagesDesign} gives
several questions that he believes must be answered before a database can be
created:

\begin{itemize}
  \item What are the entities that need to be stored by the database?
  \item What are the relationships between these entities?
  \item What constraints are there on the database?
  \item What kind of queries will be written against this database?
\end{itemize}

All the questions above are relevant during the design of the database. While
the first three questions relate to the structure of the database, the final
question is especially important regarding database performance.

His method involves drawing an entity-relationship diagram (as the name
implies, this answers the first two questions in the form of a graph) and then
translating the E-R diagram into the database schema, which includes deciding
which fields will become primary and foreign keys. Any constraints (such as
minimum or maximum lengths of strings) can then be added depending on the DBMS
product being used.

\subsection{Maintainability and the future of the software}

As the software created for this project will have to be maintained by the
University's IT Services if it is evaluated as successful, care must be taken
to ensure that the application is implemented in the most extensible and
maintainable way possible.

Principles of good software engineering apply equally to web applications as
to any other software project. There are several basic methods recommended by
Green and Ledgard \cite{Green:2011:CGF:2063166.2063168} for writing readable
and maintainable code. Their recommendations are designed around the
maintainer of the software having to do less work to understand the purpose of
a given piece of code. The recommendations could be summarised as:

\begin{itemize}
  \item Align parts of the code (e.g. equal signs) vertically when it makes sense to
  \item Write lines no longer than approximately 70 characters, and use line breaks if necessary
  \item Use simple English and short names for things that will be referenced frequently
  \item Add blank space around operators (e.g. \texttt{3 + 2 = 5} rather than \texttt{3+2=5})
  \item Ident \texttt{if} statements to allow the reader to scan the code more easily
  \item Comment code when necessary, but comments are not a solution for bad code
\end{itemize}

They note that for any system that may have a long lifespan, every effort
should be taken to improve ``readability and maintainability''. A project that
will only be used for a short amount of time by the original author does not
require as much attention to maintainability as one which will last several
years and be maintained by several different programmers.

\subsection{Sensitive information and the Data Protection Act}
\label{sec:dataprotection}

As noted in the Statement of ethics, we should discuss Charles' input here.

\subsection{Usability and user testing}

Jakob Nielsen is a web usability expert who has been publishing articles on
his website, \url{http://www.useit.com/}, since 1995. In ``Why You Only Need to
Test with 5 Users'', Nielsen asserts that usability tests should be run for all
web projects, no matter how short the project timescale or limited the budget.
This is especially relevant for a project such as the module allocation
system, where the entire application must be developed in under six months and
there is no budget allocated. Nielsen's advice is to run a single usability
test with no more than five volunteers, and to run different tests if more
participants can be recruited. His reasoning is that iterative design with
testing after each iteration will uncover any problems unwittingly created
during the development process. Finally, Nielsen points out that distinct
groups of users need to be treated separately during user testing
\cite{nielsen2000fiveusers}.

Nielsen Norman Group, a company founded by Nielsen with Don Norman in 1998,
publishes reports on web usability. Among the 230 tips offered in one such
report \cite{nng2001tipsusability}, Molich describes how to conduct user
testing sessions. The bulk of his recommendations are around making the test
participant feel comfortable during the session; this involves reassuring them
that they are not being tested, telling them that they should simply perform
the tasks as though they were at home and making the first task simple to
allow the participant to gain confidence.

Cennydd Bowles and James Box work for a web design agency based in Brighton.
In ``Undercover User Experience Design'', Bowles and Box describe various
methods of usability testing with little time or budget. They give advice on
asking questions in an unbiased way, so as not to influence the test. Like
Nielsen, they advocate around five user tests, stating that even one is better
than none. The authors suggest recording video (or, failing that, audio) of
the interview as there will not be enough time to take notes during the
session.

Bowles and Box put forward another method of eliciting information from users,
namely the corridor test. This involves watching people use the current system
for a very short amount of time and observing any usability issues they
encounter. The primary advantage of this type of test is that it takes very
little time or effort on both the part of the participant and the researcher
\cite{bowles2011undercover}.

In a 1982 paper titled ``Pitfalls of user research, and some neglected areas''
\cite{brittain1982pitfalls}, J. M. Brittain sets out the different kinds of
study that can be carried out during the research phase of any project. These
are publishing a questionnaire or interviewing users, asking users for any
input they have regarding a system or service, and observing users while they
perform a task. One point made by Brittain is that user research is
occasionally too narrow-focused - in his example, the library was focusing
``upon the demands users make for documents'' without necessarily considering
how users read the documents once they are in possession of them. In the case
of the web, one could argue that user research focuses too much on the
specific task of interest and not on how users browse the web from day-to-day
or what they generally use the web for.

\section{Development, implementation and testing}

% Several chapters describing what you have done, focusing on the novel
% aspects of your own work.

After researching the areas noted in the previous section, the software was
researched, written and tested. This section describes how users helped
influence the design of the software during interviews, a discussion on web
frameworks, the visual appearance and other important factors for web
projects.

\subsection{Database structure}

We start with a table of entities:

\begin{tabular}{ l l }
  Student    & Student ID (key), name, course \\
  Allocation & Student ID (foreign key), module ID (foreign key) \\
  Module     & Module ID (key), department ID (foreign key), name, class size \\
  Department & Department ID (key), name \\
  Staff      & Staff ID (key), department ID (foreign key) \\
\end{tabular}

As relational databases are so popular and are supported in many of the
software frameworks we look at in Section~\ref{webframeworks}, there is no
advantage to choosing a database model apart from relational.

Web frameworks abstract away anything related to a specific database product,
so the module allocation application can be written without regard for the
final database product that IT Services will use to deploy the application.

\subsubsection{Entity-relationship diagram}

% E-R diagram
% Student    ---  Allocation
% Allocation ---  Module

Figure~\ref{er_diagram} is an entity-relationship diagram of the module
allocation system:

\begin{figure}
  \begin{tikzpicture}[node distance=8em]
    \node[entity] (student) {Student};
    \node[relationship] (given) [right of=student] {given} edge (student);
    \node[entity] (allocation) [right of=given] {Allocation} edge (given);
    \node[relationship] (of) [right of=allocation] {of} edge (allocation);
    \node[entity] (module) [right of=of] {Module} edge (of);
    \node[relationship] (offers) [below of=module] {offers} edge (module);
    \node[entity] (department) [left of=offers] {Department} edge (offers);
    \node[relationship] (staffmember) [left of=department] {member} edge (department);
    \node[entity] (staff) [left of=staffmember] {Staff} edge (staffmember);
  \end{tikzpicture}
  \caption{Entity-relationship diagram for the module allocation system.}
  \label{er_diagram}
\end{figure}

\subsection{User research}

In the weeks before implementation started, students are staff were
interviewed to understand more about their relationship with module
allocation. The chart in Figure~\ref{bowles_dualpurpose_chart} was originally
published by Bowles and Box and is incredibly relevant to this project.

\begin{figure}
  \begin{center}
    \fbox{\includegraphics[width=120mm]{images/bowles_dualpurpose_testing.png}}
  \end{center}
  \caption{User session focus chart originally published in
    ``Undercover User Experience Design'', Bowles and Box \cite{bowles2011undercover}.}
  \label{bowles_dualpurpose_chart}
\end{figure}

In the meetings with students, the questions asked predominantly revolved
around module allocation the University of York; it was important to
understand how they feel about the current state of module allocation to
ensure that the system improves the experience.

Students were asked to use a mockup of the application written in \gls{html}
and the researcher observed to discover the main areas of difficulty around
using web applications. Students were asked how and where they access to web,
to get an understanding of the environments this application might be used in.

% Discuss the results of the student interviews

I also spoke to departmental administrators, who will be responsible for setting up
the system.

% Discuss the results of the staff interviews

\subsection{Web application frameworks}
\label{webframeworks}

% Intro to (web) frameworks in general

A framework is a certain amount of reusable code that helps application
developers by reducing the complexity of common web operations. For example,
any moderately complex application will need to write user input to a data
store, and protecting against malicious input is an obvious concern when
executing code in a database.

Many frameworks include functions that write to the database on behalf of the
developer, and will automatically sanitise all input to prevent against
attacks. As every application written should sanitise user input, it is this
kind of repetitive action that frameworks help ease. A useful framework should
increase the amount of time that a developer can spend building the unique
parts of their application.

While it is possible to build a web application without using a framework,
there is no advantage to rewriting basic operations for this project. If the
system needed to operate at a scale or speed far beyond what these frameworks
were capable of, it is likely that the software would have to be built from
scratch to meet those requirements.

With the limited amount of time allocated to development and implementation,
it is sensible to spend a small amount of time choosing a framework to allow
more time to implement the system.

%   Could reference the Gantt chart (if it's included) here.

% The programming language

The most important consideration when choosing a framework is the language it
is written in, as this defines, among other things, how easy the software will
be to maintain in the future.

% ColdFusion ML
%   http://www.york.ac.uk/communications/websites/content/programming/cms-integration/
% Java

There are two languages used for most web development at the University of
York. \emph{ColdFusion Markup Language} is a programming language for web
development, the most popular commercial implementation of which is Adobe
ColdFusion. ColdFusion is used widely across the University, including
applications such as the ``People Directory'' search tool and account
management facilities offered by IT Services. \emph{Java} was used in the
creation of the new Student Portal.

In a presentation at \gls{oscon} in 2007 \cite{raible2007javawebframeworks},
Matt Raible compared several web frameworks written in Java; JavaServer Faces
(JSF), Spring MVC, Stripes, Struts 2, Tapestry and Wicket.

% JSF: Fast and easy
% Stripes: Convention over config, good docs, 
% Struts: poor docs
% Tapestry: templates are HTML, steep learning curve
% Wicket, great for Java developers, not web developers

% Ruby
% Python
% PHP

Outside of the University of York, popular web programming languages include
\emph{Ruby}, \emph{Python} and \emph{PHP}.

I chose a framework. It could be based on ColdFusion, Ruby or Python. Or
something else entirely, even PHP or Java. But at the moment I'm not sure.

Rails guesses the model's attributes based on the database schema, whereas
Django requires that the model's attributes be listed in the application, and
the schema is then created.

\begin{lstlisting}
class Allocation < ActiveRecord::Base
  belongs_to :student
  belongs_to :module
end
\end{lstlisting}

There's a good comparison of Django and Rails which is quite methodical. In
it, the authors note that equivalent applications took $\frac{2}{3}$ the time
to be implemented in Django than Rails \cite{RailsDjangoComparison_2007}.

There's also a good comparison of the three big players that pretty much rules
out CakePHP as being rubbish \cite{EvalWebDevFrameworks_2009}.

\subsection{Visual appearance}

The application should be visually consistent with other University web
software to instill trust in the user.

\begin{figure}
  \begin{center}
    \includegraphics[width=160mm]{images/2011_11_06_yorkacuk.png}
  \end{center}
  \caption{A general page on the University site, using University colours.}
  \label{yorkacuk_general_page}
\end{figure}

\begin{figure}
  \begin{center}
    \includegraphics[width=160mm]{images/2011_11_06_yorkacuk_directory.png}
  \end{center}
  \caption{The University of York People Directory search tool.}
  \label{yorkacuk_directory_search}
\end{figure}

% Include screenshots of University software:
%   https://www.york.ac.uk/it-services/facilities/account/accounts
%   https://www.cs.york.ac.uk/submit/assessment/

\subsection{Accessibility}

Perhaps talking about Dave Swallow, CS PhD.

\subsection{Allocation algorithm}

Let's discuss the algorithm here.

\subsection{Testing}

How will the software be tested?

\subsection{Pilot by the Archaeology and History departments}

How did the pilot go?

\section{Further work}

% A chapter describing possible ways in which your work could be continued or
% developed. Be imaginative but realistic.

We should probably ask departments what they'd like to see in the future.

If I didn't have time to implement a weighting system (rather than a simple,
plain ranking system) I could talk about that here.

\section{Conclusions}

% This is similar to the abstract. The difference is that you should assume
% here that the reader of the conclusions has read the rest of the report.

It was a success. Well, hopefully.

\appendix

% What should go in an appendix? Screenshots? Code? Something else?

\newpage
\section{External links}

This appendix gives URLs to departments, offices or services mentioned
throughout the document.

\url{http://www.york.ac.uk/about/organisation/governance/sub-committees/teaching-committee/}

\url{http://www.york.ac.uk/it-services/}

I would like to make the code of this software available online, at
\url{https://github.com/alexmuller/york-allocation}.

\newpage
\section{Participant consent form}
\label{sec:consent}

All research participants (students and staff of the University of York)
signed the consent form shown in Figure~\ref{participantconsent} immediately
before their interview took place. This consent form is adapted from one made
available by Alistair Edwards for Computer Science students to use during
their projects. As of 5 November 2011, the original is available at
\url{http://www-users.cs.york.ac.uk/~alistair/projects/consent.html}.

In each case, the top half of the form was retained by the project author and
the second half was given to the participant in case they had any further
questions about the interview.

\begin{figure}[h]
  \begin{center}
    \fbox{\includegraphics[width=140mm, trim=0 80mm 0 0]{images/consent.pdf}}
  \end{center}
  \caption{Participant consent form.}
  \label{participantconsent}
\end{figure}

\newpage
\printglossaries

\newpage
\bibliographystyle{references/IEEEtran.bst} % not just plain
\bibliography{references/references.bib}


\end{document}
